% MIT License
%
% Copyright (c) 2018 Jelle Spijker
%
% Permission is hereby granted, free of charge, to any person obtaining a copy
% of this software and associated documentation files (the "Software"), to deal
% in the Software without restriction, including without limitation the rights
% to use, copy, modify, merge, publish, distribute, sublicense, and/or sell
% copies of the Software, and to permit persons to whom the Software is
% furnished to do so, subject to the following conditions:
%
% The above copyright notice and this permission notice shall be included in all
% copies or substantial portions of the Software.
%
% THE SOFTWARE IS PROVIDED "AS IS", WITHOUT WARRANTY OF ANY KIND, EXPRESS OR
% IMPLIED, INCLUDING BUT NOT LIMITED TO THE WARRANTIES OF MERCHANTABILITY,
% FITNESS FOR A PARTICULAR PURPOSE AND NONINFRINGEMENT. IN NO EVENT SHALL THE
% AUTHORS OR COPYRIGHT HOLDERS BE LIABLE FOR ANY CLAIM, DAMAGES OR OTHER
% LIABILITY, WHETHER IN AN ACTION OF CONTRACT, TORT OR OTHERWISE, ARISING FROM,
% OUT OF OR IN CONNECTION WITH THE SOFTWARE OR THE USE OR OTHER DEALINGS IN THE
% SOFTWARE.

\usepackage[per-mode=fraction,
 separate-uncertainty=true,
 bracket-unit-denominator=false,
 multi-part-units=single]{siunitx}

\usetikzlibrary{circuits}
\usepgflibrary{shapes.gates.pid}

% Setting up an pid circuit
\tikzset{
 circuit pid/.style={
   circuit,
   every circuit pid/.try
  }
}

% Symbols user-friendly aliases
\tikzset{
 pump/.style={2301={#1}},
 compressor/.style={2302={#1}},
 fan/.style={2302={#1}},
 valve/.style={2101={#1}},
 three way valve/.style={2103={#1}},
 measurement point/.style={1011={#1}},
 measurement device/.style={1041={#1}},
 manual operation/.style={1021={#1}},
 automatic operation/.style={1022={#1}},
 turning actuator/.style={685={#1}},
 envelope/.style={301={#1}},
 tank/.style={301={#1}},
 boiler/.style={2531={#1}},
 steam generator/.style={2531={#1}},
}

% Annotations user-friendly aliases
\tikzset{
 displacement/.style={2321},
 rotodynamic/.style={2322},
 centrifugal/.style={2322},
 non return/.style={2111},
 central control room/.style={1101},
 local control room/.style={1103},
 heating coil/.style={2501},
 cooling coil/.style={2501},
 screen element/.style={2602},
 filter element/.style={2602},
 spray nozzle/.style={2037},
 fired type/.style={2541},
}

% The default symbols (you need to load a sublib to install the actual rendering).
\tikzset{
 circuit declare symbol = 2301, % Liquid pump
 circuit declare symbol = 2302, % Gas pump
 circuit declare symbol = 2101, % Two-way valve
 circuit declare symbol = 2103, % Three-way valve
 circuit declare symbol = 1011, % measurement point
 circuit declare symbol = 1041, % Information-processing function
 circuit declare symbol = 1021, % Manual operation of a final controlling element
 circuit declare symbol = 1022, % Automatic operation of a final controlling element
 circuit declare symbol = 685, % Manual actuator operated by turning
 circuit declare symbol = 301, % Envelope (tank)
 circuit declare symbol = 2501, % Heating or cooling coil
 circuit declare symbol = 2602, % Screen or filter element
 circuit declare symbol = 2037, % Spray nozzle
 circuit declare symbol = 2541, % Fired type
 circuit declare symbol = 2531, % Boiler, steam generator
 circuit declare symbol = flow direction,
 circuit declare symbol = flow direction',
 circuit declare symbol = branch,
 set flow direction graphic  = flow direction pid graphic,
 set flow direction' graphic = flow direction' pid graphic,
}

% The default units
\tikzset{
 circuit declare unit={speed}{\si{\meter\per\second}},
}

% The direction and arrow settings
\tikzset{
 annotation arrow/.style = {
   /utils/exec={\pgfsetarrowoptions{direction pid}{.4*\the\tikzcircuitssizeunit+.3*\the\pgflinewidth}},
   >=direction pid
  }
}

\tikzset{
 % These styles should set the end-arrow.
 %
 % This arrow will generally be used to indicate flow directions in a circuit:
 flow direction arrow/.style = {
   /utils/exec={\pgfsetarrowoptions{direction pid}{1.3065*.5*\the\tikzcircuitssizeunit+1.3065*.3*\the\pgflinewidth}},
   >=direction pid, direction pid arrow = direction pid,
  }
}

\tikzset{
flow direction pid graphic/.style = {
  shape=direction pid,
  circuit symbol filled,
  flow direction arrow,
  minimum width  = .5*\the\tikzcircuitssizeunit+.3*\the\pgflinewidth,
  minimum height = .5*\the\tikzcircuitssizeunit+.3*\the\pgflinewidth,
  transform shape
 },
flow direction' pid graphic/.style = {
flow direction pid graphic,
rotate=180
}
}

% Process measure and control
\tikzset{
 measure/.style args={#1}{
   append after command={%
     \pgfextra
     \node[scale=0.4, anchor=south, yshift=-2*\the\pgflinewidth] at (\tikzlastnode) {#1};
     \tikz@after@path%
     \endpgfextra
    }
  },
 at/.style args={#1}{
   above=0.25 of #1,
   prefix after command={%
     \pgfextra
     \draw (\tikzlastnode) to (#1.center);
     \tikz@after@path%
     \endpgfextra
    }
  },
 with/.style args={#1#2#3}{
   append after command={%
     \pgfextra
     \node[#1] at ([xshift=#2*\tikzcircuitssizeunit,yshift=#3*\tikzcircuitssizeunit]\tikzlastnode) (\tikzlastnode-#1) {};
     \tikz@after@path%
     \endpgfextra
    }
  }
}

% Annotations
\tikzset{
 circuit declare annotation={2321}{0pt} % Positive-displacement type
  {
   (\tikzlastnode.north) edge[-,line width=0.5pt,line to] (\tikzlastnode.south)
  },
 circuit declare annotation={2322}{0pt} % Rotodynamic type
  {
   (\tikzlastnode.west) edge[-.,line width=0.5pt,line to] (\tikzlastnode.east)
  },
 circuit declare annotation={2111}{0pt} % Non-return function Flow from left to right possible
  {
   ([xshift=-1.5pt,yshift=-1pt]\tikzlastnode.north west) edge[-.,line width=1.5pt,to path={arc (-360:0:0.75pt)}] ()
  },
 circuit declare annotation={1101}{0pt} % Primary location in a central control room
  {
   (\tikzlastnode.east) edge[-,line width=0.5pt,line to] (\tikzlastnode.west)
  },
 circuit declare annotation={1103}{0pt} % Primary location in a local control room
  {
   ([xshift=-\the\pgflinewidth]\tikzlastnode.east) edge[double, -,line width=0.5pt,line to] ([xshift=\the\pgflinewidth]\tikzlastnode.west)
  },
 circuit declare annotation={adjustable}{1.5\tikzcircuitssizeunit}
  {
   [shift=(\tikzlastnode.center)]
   (-1.5\tikzcircuitssizeunit,-1.5\tikzcircuitssizeunit) edge[line to] (1.5\tikzcircuitssizeunit,1.5\tikzcircuitssizeunit)
  },
 circuit declare annotation={adjustable'}{1.5\tikzcircuitssizeunit}
  {
   [shift=(\tikzlastnode.center)]
   (-1.5\tikzcircuitssizeunit,1.5\tikzcircuitssizeunit) edge[line to] (1.5\tikzcircuitssizeunit,-1.5\tikzcircuitssizeunit)
  },
}

% Helper command to get the current coordinates in a path
\makeatletter
\newcommand\currentcoordinate{\the\tikz@lastxsaved,\the\tikz@lastysaved}
\makeatother

\endinput
