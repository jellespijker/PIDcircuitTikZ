% MIT License
%
% Copyright (c) 2018 Jelle Spijker
%
% Permission is hereby granted, free of charge, to any person obtaining a copy
% of this software and associated documentation files (the "Software"), to deal
% in the Software without restriction, including without limitation the rights
% to use, copy, modify, merge, publish, distribute, sublicense, and/or sell
% copies of the Software, and to permit persons to whom the Software is
% furnished to do so, subject to the following conditions:
%
% The above copyright notice and this permission notice shall be included in all
% copies or substantial portions of the Software.
%
% THE SOFTWARE IS PROVIDED "AS IS", WITHOUT WARRANTY OF ANY KIND, EXPRESS OR
% IMPLIED, INCLUDING BUT NOT LIMITED TO THE WARRANTIES OF MERCHANTABILITY,
% FITNESS FOR A PARTICULAR PURPOSE AND NONINFRINGEMENT. IN NO EVENT SHALL THE
% AUTHORS OR COPYRIGHT HOLDERS BE LIABLE FOR ANY CLAIM, DAMAGES OR OTHER
% LIABILITY, WHETHER IN AN ACTION OF CONTRACT, TORT OR OTHERWISE, ARISING FROM,
% OUT OF OR IN CONNECTION WITH THE SOFTWARE OR THE USE OR OTHER DEALINGS IN THE
% SOFTWARE.

\usepackage[per-mode=fraction,
 separate-uncertainty=true,
 bracket-unit-denominator=false,
 multi-part-units=single]{siunitx}
\usepackage{ifthen}
\usepackage{xparse}

\usetikzlibrary{circuits}
\usetikzlibrary{decorations.markings}
\usetikzlibrary{decorations.pathreplacing}
\usetikzlibrary{intersections}
\usepgflibrary{shapes.gates.pid}

% Setting up an pid circuit
\tikzset{
 circuit pid/.style={
   circuit,
   every circuit pid/.try
  }
}

% Symbols user-friendly aliases
\tikzset{
 pump/.style={2301={#1}},
 compressor/.style={2302={#1}},
 fan/.style={2302={#1}},
 valve/.style={2101={#1}},
 angled valve/.style={2102={#1}},
 three way valve/.style={2103={#1}},
 four way valve/.style={2104={#1}},
 damper/.style={2151={#1}},
 measurement point/.style={1011={#1}},
 measurement device/.style={1041={#1}},
 manual operation/.style={1021={#1}},
 automatic operation/.style={1022={#1}},
 turning actuator/.style={685={#1}},
 spring/.style={2002={#1}},
 envelope/.style={301={#1}},
 tank/.style={301={#1}},
 tank atmospheric pressure/.style={2061={#1}},
 container/.style={2061={#1}},
 cistern/.style={2061={#1}},
 boiler/.style={2531={#1}},
 steam generator/.style={2531={#1}},
 heat pump/.style={2551={#1}},
 refrigerator/.style={2551={#1}},
 freezer/.style={2551={#1}},
 pressure vessel/.style={2062={#1}},
 vacuum vessel/.style={2062={#1}},
 regenerative pre-heater/.style={2518={#1}},
 straight tube heat exchanger/.style={2511={#1}},
 cooling tower/.style={2521={#1}},
 coil shaped heat exchanger/.style={2514={#1}},
}

% Annotations user-friendly aliases
\tikzset{
 displacement/.style={2321},
 rotodynamic/.style={2322},
 centrifugal/.style={2322},
 non return/.style={2111},
 globe/.style={2121},
 safety function/.style={2112},
 central control room/.style={1101},
 local control room/.style={1103},
 heating coil/.style={2501={#1}},
 cooling coil/.style={2501={#1}},
 screen element/.style={2602={#1}},
 filter element/.style={2602={#1}},
 bag filter element/.style={2602-bag={#1}},
 spray nozzle/.style={2037={#1}},
 fired type/.style={2541={#1}},
 quantity transition to higher/.style={130},
 quantity transition to lower/.style={131},
 quantity transition to higher and lower/.style={132},
 device for separating/.style={2601},
}

% Path style user-friendly aliases
\tikzset{
 functional path/.style={431},
 analogue signal path/.style={234},
}

% The default symbols (you need to load a sublib to install the actual rendering).
\tikzset{
 circuit declare symbol = 2301, % Liquid pump
 circuit declare symbol = 2302, % Gas pump
 circuit declare symbol = 2101, % Two-way valve
 circuit declare symbol = 2102, % Angled two-way valve
 circuit declare symbol = 2103, % Three-way valve
 circuit declare symbol = 2104, % Four-way valve
 circuit declare symbol = 2151, % Two- or three-way damper
 circuit declare symbol = 1011, % measurement point
 circuit declare symbol = 1041, % Information-processing function
 circuit declare symbol = 1021, % Manual operation of a final controlling element
 circuit declare symbol = 1022, % Automatic operation of a final controlling element
 circuit declare symbol = 685, % Manual actuator operated by turning
 circuit declare symbol = 2002, % Spring
 circuit declare symbol = 301, % Envelope (tank)
 circuit declare symbol = 2061, % Container tank, cistern for atmospheric pressure
 circuit declare symbol = 2501, % Heating or cooling coil
 circuit declare symbol = 2602, % Screen or filter element
 circuit declare symbol = 2602-bag, % Bag screen or filter element
 circuit declare symbol = 2037, % Spray nozzle
 circuit declare symbol = 2541, % Fired type
 circuit declare symbol = 2531, % Boiler, steam generator
 circuit declare symbol = 2551, % Heat pump, refrigerator or freezer
 circuit declare symbol = 2062, % Pressure or vacuum vessel
 circuit declare symbol = 2518, % Regenerative pre-heater
 circuit declare symbol = 2511, % Heat-exchanger with straight tubes (fixed-tube plates)
 circuit declare symbol = 2514, % Heat-exchanger with coil-shaped tubes
 circuit declare symbol = 2521, % Cooling tower
 circuit declare symbol = flow direction,
 circuit declare symbol = flow direction',
 circuit declare symbol = branch,
 circuit declare symbol = crossing,
 circuit declare symbol = reference,
 set flow direction graphic  = flow direction pid graphic,
 set flow direction' graphic = flow direction' pid graphic,
}

% The default units
\tikzset{
 circuit declare unit={speed}{\si{\meter\per\second}},
 circuit declare unit={mass flow}{\si{\kilo\gram\per\second}},
 circuit declare unit={volume flow}{\si{\cubic\meter\per\second}},
}

% The direction and arrow settings
\tikzset{
 annotation arrow/.style = {
   /utils/exec={\pgfsetarrowoptions{direction pid}{.25*\the\tikzcircuitssizeunit+.2*\the\pgflinewidth}},
   >=direction pid
  }
}

\tikzset{
 % These styles should set the end-arrow.
 %
 % This arrow will generally be used to indicate flow directions in a circuit:
 flow direction arrow/.style = {
   /utils/exec={\pgfsetarrowoptions{direction pid}{1.3065*.5*\the\tikzcircuitssizeunit+1.3065*.3*\the\pgflinewidth}},
   >=direction pid, direction pid arrow = direction pid,
  }
}

\tikzset{
flow direction pid graphic/.style = {
  shape=direction pid,
  circuit symbol filled,
  flow direction arrow,
  minimum width  = .5*\the\tikzcircuitssizeunit+.3*\the\pgflinewidth,
  minimum height = .5*\the\tikzcircuitssizeunit+.3*\the\pgflinewidth,
  transform shape
 },
flow direction' pid graphic/.style = {
flow direction pid graphic,
rotate=180
}
}

% Process measure and control
\tikzset{
 measure/.style args={#1}{
   append after command={%
     \pgfextra
     \node[scale=0.35, anchor=south, yshift=-2*\the\pgflinewidth] at (\tikzlastnode) {#1};
     \tikz@after@path%
     \endpgfextra
    }
  },
 at/.style args={#1#2#3}{
   above=#2\tikzcircuitssizeunit of #1, #3,
   append after command={%
     \pgfextra{
      \draw (\tikzlastnode) to (#1);}
    }
  },
 with/.style args={#1#2#3#4}{
   append after command={%
     \pgfextra{
      \node[#1={name=\tikzlastnode-#1,#4}] at ([xshift=#2*\tikzcircuitssizeunit,yshift=#3*\tikzcircuitssizeunit]\tikzlastnode) {};}
    }
  }
}

\newif\ifshowid\showidfalse
\newif\ifshowidinv\showidinvfalse

\tikzset{
 id/.store in=\flowid,
 id=001,
 pos/.store in=\flowpos,
 pos=0.5,
 /tikz/show/.is if=showid,
 /tikz/show'/.is if=showidinv,
 /tikz/show=false,
 /tikz/show'=false,
 flow path/.style={
   name path global=\fsid\flowid,
   /utils/exec=\csname tikz@options\endcsname,
   decoration={
     markings,
     path has corners,
     mark=at position \flowpos
     with {
       \ifshowid \node[flow direction={info sloped=\fsid\flowid}] {}; \else \node[flow direction] {}; \fi
       \ifshowidinv \node[flow direction={info sloped={[rotate=-180,label distance=.5\tikzcircuitssizeunit]\fsid\flowid}}] {}; \else \node[flow direction] {}; \fi}
    },
   postaction={decorate}
  },
 431/.style={
   decoration={
     markings,
     mark=between positions 0 and 1 step 0.2 with {
       \draw (-0.25\tikzcircuitssizeunit,-0.25\tikzcircuitssizeunit) to (0.25\tikzcircuitssizeunit,0.25\tikzcircuitssizeunit);
      }
    },
   postaction={decorate}
  },
 234/.style={
   decoration={
     markings,
     mark=between positions 0.2 and 0.8 step 0.2 with {
       \draw (-0.25\tikzcircuitssizeunit,-0.25\tikzcircuitssizeunit) to (-0.25\tikzcircuitssizeunit,0.125\tikzcircuitssizeunit)
       arc (180:0:0.25\tikzcircuitssizeunit) to (0.25\tikzcircuitssizeunit,-0.25\tikzcircuitssizeunit);
      }
    },
   postaction={decorate}
  },
}


% Annotations
\tikzset{
 circuit declare annotation={2321}{0pt} % Positive-displacement type
  {
   (\tikzlastnode.north) edge[-,line width=0.5pt,line to] (\tikzlastnode.south)
  },
 circuit declare annotation={2322}{0pt} % Rotodynamic type
  {
   (\tikzlastnode.west) edge[-.,line width=0.5pt,line to] (\tikzlastnode.east)
  },
 circuit declare annotation={2111}{0pt} % Non-return function Flow from left to right possible
  {
   ([yshift=-0.6\tikzcircuitssizeunit,xshift=0.2\tikzcircuitssizeunit]\tikzlastnode.north west) edge[-,line width=2pt,to path={arc (-360:0:1pt)}] ()
  },
 circuit declare annotation={1101}{0pt} % Primary location in a central control room
  {
   (\tikzlastnode.east) edge[-,line width=0.5pt,line to] (\tikzlastnode.west)
  },
 circuit declare annotation={1103}{0pt} % Primary location in a local control room
  {
   ([xshift=-\the\pgflinewidth]\tikzlastnode.east) edge[double, -,line width=0.5pt,line to] ([xshift=\the\pgflinewidth]\tikzlastnode.west)
  },
 circuit declare annotation={adjustable}{1.5\tikzcircuitssizeunit}
  {
   ([xshift=-0.25\tikzcircuitssizeunit,yshift=-0.25\tikzcircuitssizeunit]\tikzlastnode.south west) edge[line to] ([xshift=0.25\tikzcircuitssizeunit, yshift=0.25\tikzcircuitssizeunit]\tikzlastnode.north east)
  },
 circuit declare annotation={adjustable'}{1.5\tikzcircuitssizeunit}
  {
   ([xshift=-0.25\tikzcircuitssizeunit,yshift=0.25\tikzcircuitssizeunit]\tikzlastnode.north west) edge[line to] ([xshift=0.25\tikzcircuitssizeunit, yshift=-0.25\tikzcircuitssizeunit]\tikzlastnode.south east)
  },
 circuit declare annotation={130}{0pt} % Transition to a higher quantity level
  {
   [shift=(\tikzlastnode.center),xshift=0.3\tikzcircuitssizeunit, yshift=0.5\tikzcircuitssizeunit]
   (0, -0.25\tikzcircuitssizeunit) edge[line to] (0,0.5\tikzcircuitssizeunit)
  },
 circuit declare annotation={131}{0pt} % Transition to a lower quantity level
  {
   [shift=(\tikzlastnode.center),xshift=0.3\tikzcircuitssizeunit, yshift=0.5\tikzcircuitssizeunit]
   (0,0.5\tikzcircuitssizeunit) edge[line to] (0, -0.25\tikzcircuitssizeunit)
  },
 circuit declare annotation={132}{0pt} % Transition to a higher or lower quantity level
  {
   [shift=(\tikzlastnode.center),xshift=0.3\tikzcircuitssizeunit, yshift=0.5\tikzcircuitssizeunit]
   (0,0.5\tikzcircuitssizeunit) edge[<->,line to] (0, -0.25\tikzcircuitssizeunit)
  },
 circuit declare annotation={2112}{0pt} % Safety function
  {
   [shift=(\tikzlastnode.north east),xshift=-0.3\tikzcircuitssizeunit, yshift=-0.3\tikzcircuitssizeunit]
   (0,0) edge[-,line to] ([xshift=-0.3\tikzcircuitssizeunit, yshift=0.3\tikzcircuitssizeunit]\tikzlastnode.south east)
  },
 circuit declare annotation={2121}{0pt} % Globe type
  {
   ([xshift=2pt]\tikzlastnode.center) edge[-,line width=4pt,to path={arc (-360:0:2pt)}] ()
  },
 circuit declare annotation={light}{0pt} % N
  {
   ([xshift=-0.75\tikzcircuitssizeunit,yshift=1\tikzcircuitssizeunit]\tikzlastnode.north) edge[line to] ([xshift=-0.25\tikzcircuitssizeunit,yshift=0.1\tikzcircuitssizeunit]\tikzlastnode.north)
   ([xshift=-0.25\tikzcircuitssizeunit,yshift=1\tikzcircuitssizeunit]\tikzlastnode.north) edge[line to] ([xshift=0.25\tikzcircuitssizeunit,yshift=0.1\tikzcircuitssizeunit]\tikzlastnode.north)
  },
 circuit declare annotation={2601}{0pt} % N
  {
   ([xshift=-0.75\tikzcircuitssizeunit,yshift=0.75\tikzcircuitssizeunit]\tikzlastnode.center) edge[line to] ([xshift=0.75\tikzcircuitssizeunit,yshift=0.75\tikzcircuitssizeunit]\tikzlastnode.center)
   ([xshift=-0.25\tikzcircuitssizeunit,yshift=0.75\tikzcircuitssizeunit]\tikzlastnode.center) edge[line to] ([xshift=0.65\tikzcircuitssizeunit,yshift=0.35\tikzcircuitssizeunit]\tikzlastnode.center)
  },
}

\makeatletter
% Helper command to get the current coordinates in a path
\newcommand\currentcoordinate{\the\tikz@lastxsaved,\the\tikz@lastysaved}

% Helper to place crossing
\newcommand{\crossings}[2]{
 \fill [name intersections={of=#1 and #2, name=i, total=\t}]
 \foreach \s in {1,...,\t}{(i-\s) node[crossing] {}};
}

% flow system environment
\pgfkeys{
 flow system/.is family, /flow system,
 fsdefault/.style =
  {color=black, id=0, line type=solid, line width=0.25mm},
 color/.estore in = \fscolor,
 id/.estore in = \fsid,
 line type/.estore in = \fslinetype,
 line width/.estore in = \fslinewidth,
}

\NewDocumentEnvironment{flow system} { o }
{
 \pgfkeys{/flow system, fsdefault, #1}
 \begin{scope}[\fscolor, line width=\fslinewidth, \fslinetype]
  }
  {
 \end{scope}
}

% complex device environment
\newif\ifshowcdlabel\showcdlabelfalse
\newif\ifshowcdborder\showcdborderfalse

\pgfkeys{
 complex device/.is family, /complex device,
 cddefault/.style =
  {name=device,xshift=0,yshift=0,rotate=0, border=5, border line type=dash dot, border line width=0.25mm, border color=black},
 xshift/.estore in = \cdxshift,
 yshift/.estore in = \cdyshift,
 rotate/.estore in = \cdrotate,
 name/.estore in = \cdname,
 show label/.is if = showcdlabel,
 show border/.is if = showcdborder,
 border/.estore in = \cdborder,
 border line type/.estore in = \cdborderlinetype,
 border line width/.estore in = \cdborderlinewidth,
 border color/.estore in = \cdbordercolor,
}

\NewDocumentEnvironment{complex device} { o }
{
 \pgfkeys{/complex device, cddefault, #1}
 \begin{scope}[xshift=\cdxshift, yshift=\cdyshift, rotate=\cdrotate, local bounding box=\cdname]
  }
  {
  \ifshowcdborder \draw[\cdbordercolor, \cdborderlinetype, line width=\cdborderlinewidth] ([xshift=-\cdborder,yshift=\cdborder]\cdname.north west) rectangle ([xshift=\cdborder,yshift=-\cdborder]\cdname.south east); \fi
  \ifshowcdlabel \node at ([yshift=\cdborder+4]\cdname.north) {\cdname}; \fi
 \end{scope}
}
\makeatother

\endinput
