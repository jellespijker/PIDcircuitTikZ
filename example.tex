% MIT License
%
% Copyright (c) 2018 Jelle Spijker
%
% Permission is hereby granted, free of charge, to any person obtaining a copy
% of this software and associated documentation files (the "Software"), to deal
% in the Software without restriction, including without limitation the rights
% to use, copy, modify, merge, publish, distribute, sublicense, and/or sell
% copies of the Software, and to permit persons to whom the Software is
% furnished to do so, subject to the following conditions:
%
% The above copyright notice and this permission notice shall be included in all
% copies or substantial portions of the Software.
%
% THE SOFTWARE IS PROVIDED "AS IS", WITHOUT WARRANTY OF ANY KIND, EXPRESS OR
% IMPLIED, INCLUDING BUT NOT LIMITED TO THE WARRANTIES OF MERCHANTABILITY,
% FITNESS FOR A PARTICULAR PURPOSE AND NONINFRINGEMENT. IN NO EVENT SHALL THE
% AUTHORS OR COPYRIGHT HOLDERS BE LIABLE FOR ANY CLAIM, DAMAGES OR OTHER
% LIABILITY, WHETHER IN AN ACTION OF CONTRACT, TORT OR OTHERWISE, ARISING FROM,
% OUT OF OR IN CONNECTION WITH THE SOFTWARE OR THE USE OR OTHER DEALINGS IN THE
% SOFTWARE.

\documentclass[a4paper]{article}
\usepackage{listings}
\usepackage{xcolor,showexpl}

\usepackage{tikz}
\usetikzlibrary{circuits}
\usetikzlibrary{circuits.pid.ISO14617}
\usetikzlibrary{positioning,calc}

\lstdefinestyle{circuitpid}{moretexcs={draw, node, usetikzlibrary},
backgroundcolor=\color{white}, basicstyle=\footnotesize, breakatwhitespace=false,
breaklines=true, captionpos=b, commentstyle=\color{green}, escapeinside={\%*}{*)},
extendedchars=true, keepspaces=true, keywordstyle=\color{blue}, language=[LaTeX]Tex,
numbers=left, numbersep=1pt, numberstyle=\tiny\color{gray}, showspaces=false,
showstringspaces=false, showtabs=false, stringstyle=\color{mauve}, tabsize=2,
preset=\centering, pos=b
}

\begin{document}
\def\ {}
\title{Tikz P\&ID circuit extension}
\author{Jelle Spijker}
\date{\today}
\maketitle


\begin{LTXexample}[style=circuitpid, caption=P\&ID example code]
 \usepackage{tikz}
 \usetikzlibrary{circuits}
 \usetikzlibrary{circuits.pid.ISO14617}
 \usetikzlibrary{positioning,calc}

 \begin{tikzpicture}[
   circuit pid ISO14617,
   every info/.style={font=\tiny}]

  \draw (0,0) to [pump={displacement,name=P1,info=$P_1$}] (2,0)
  to [branch={name=T1}] (2.5,0)
  to [flow direction={speed=3}] (3,0)
  to [valve={name=V1,info'=$V_{1}$}](4,0)
  to [three way valve={name=V2,info=below right:$V_2$}] ++(1,0)
  to [tank={name=B1,with={heating coil}{0pt}{0pt}}] ++(1,0)
  to [tank={filter element, name=F1}] ++ (1,0);
  \draw (V2.south) to [pump={name=P2,info=$P_2$}] ++(0,-2)
  to [measurement point={name=M1}] ++(-2,0)
  to (\currentcoordinate -| T1)
  to [valve={non return,info=$V_3$}] (T1);
  \node[measurement device=local control room, at=M1, measure=P]{};
  \node[turning actuator, at=V1]{};
  \node[automatic operation, at=V2]{};
  \draw (B1-heating coil.south) to ++(0,-0.5)
  to [valve, tiny circuit symbols] ++(1, 0);
  \draw (B1-heating coil.north) to ++(0,0.5)
  to ++(1,0);


 \end{tikzpicture}
\end{LTXexample}

\end{document}
