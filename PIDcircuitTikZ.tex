% MIT License
%
% Copyright (c) 2018 Jelle Spijker
%
% Permission is hereby granted, free of charge, to any person obtaining a copy
% of this software and associated documentation files (the "Software"), to deal
% in the Software without restriction, including without limitation the rights
% to use, copy, modify, merge, publish, distribute, sublicense, and/or sell
% copies of the Software, and to permit persons to whom the Software is
% furnished to do so, subject to the following conditions:
%
% The above copyright notice and this permission notice shall be included in all
% copies or substantial portions of the Software.
%
% THE SOFTWARE IS PROVIDED "AS IS", WITHOUT WARRANTY OF ANY KIND, EXPRESS OR
% IMPLIED, INCLUDING BUT NOT LIMITED TO THE WARRANTIES OF MERCHANTABILITY,
% FITNESS FOR A PARTICULAR PURPOSE AND NONINFRINGEMENT. IN NO EVENT SHALL THE
% AUTHORS OR COPYRIGHT HOLDERS BE LIABLE FOR ANY CLAIM, DAMAGES OR OTHER
% LIABILITY, WHETHER IN AN ACTION OF CONTRACT, TORT OR OTHERWISE, ARISING FROM,
% OUT OF OR IN CONNECTION WITH THE SOFTWARE OR THE USE OR OTHER DEALINGS IN THE
% SOFTWARE.
\pdfminorversion=5
\pdfobjcompresslevel=2
\documentclass[a4paper]{ltxdoc}
\usepackage[hyphens]{url}
\usepackage[pdfborder={0 0 0}]{hyperref}
\usepackage[version=latest]{pgf}
\usepackage{calc,listings,tikz,units}

\usepackage{hyperref}
\hypersetup{%
	colorlinks=true, % use true to enable colors below:
	linkcolor=blue,%red,
	filecolor=blue,%magenta,
	urlcolor=blue,%cyan,
	citecolor=blue,
	pdfborder=0 0 0,
}
\usepackage{tikz}
\usetikzlibrary{backgrounds, circuits.pid.ISO14617}

\newif\ifgdccodebasic
\newif\ifgdccodeogdf

\usepackage[a4paper,left=2.25cm,right=2.25cm,top=2.5cm,bottom=2.5cm,nohead]{geometry}
\usepackage{amsmath,amssymb}
\usepackage{xxcolor}
\usepackage{makeidx}
\usepackage{enumitem}
\usepackage[T1]{fontenc}
\usepackage[latin9]{inputenc}
\usepackage{siunitx}
\usepackage{tabu}

% Copyright 2006 by Till Tantau
%
% This file may be distributed and/or modified
%
% 1. under the LaTeX Project Public License and/or
% 2. under the GNU Free Documentation License.
%
% See the file doc/generic/pgf/licenses/LICENSE for more details.

% $Header: /cvsroot/pgf/pgf/doc/generic/pgf/macros/pgfmanual-en-macros.tex,v 1.81 2014/03/20 10:07:44 tantau Exp $


\providecommand\href[2]{\texttt{#1}}
\providecommand\hypertarget[2]{\texttt{#1}}
\providecommand\hyperlink[2]{\texttt{#1}}


\colorlet{examplefill}{yellow!80!black}
\definecolor{graphicbackground}{rgb}{0.96,0.96,0.8}
\definecolor{codebackground}{rgb}{0.9,0.9,1}

\newenvironment{pgfmanualentry}{\list{}{\leftmargin=2em\itemindent-\leftmargin\def\makelabel##1{\hss##1}}}{\endlist}
\newcommand\pgfmanualentryheadline[1]{\itemsep=0pt\parskip=0pt{\raggedright\item\strut{#1}\par}\topsep=0pt}
\newcommand\pgfmanualbody{\parskip3pt}

\let\origtexttt=\texttt
\def\texttt#1{{\def\textunderscore{\char`\_}\def\textbraceleft{\char`\{}\def\textbraceright{\char`\}}\origtexttt{#1}}}
\def\exclamationmarktext{!}
\def\atmarktext{@}

{
  \catcode`\|=12
  \gdef\pgfmanualnormalbar{|}
  \catcode`\|=13
  \AtBeginDocument{\gdef|{\ifmmode\pgfmanualnormalbar\else\expandafter\verb\expandafter|\fi}}
}



\newenvironment{pgflayout}[1]{
  \begin{pgfmanualentry}
    \pgfmanualentryheadline{%
      \pgfmanualpdflabel{#1}{}%
      \texttt{\string\pgfpagesuselayout\char`\{\declare{#1}\char`\}}\oarg{options}%
    }
    \index{#1@\protect\texttt{#1} layout}%
    \index{Page layouts!#1@\protect\texttt{#1}}%
    \pgfmanualbody
}
{
  \end{pgfmanualentry}
}


\newenvironment{command}[1]{
  \begin{pgfmanualentry}
    \extractcommand#1\@@
    \pgfmanualbody
}
{
  \end{pgfmanualentry}
}

\makeatletter

\def\includeluadocumentationof#1{
  \directlua{require 'pgf.manual.DocumentParser'}
  \directlua{pgf.manual.DocumentParser.include '#1'}
}

\newenvironment{luageneric}[4]{
  \pgfmanualentry
    \pgfmanualentryheadline{#4 \texttt{#1\declare{#2}}#3}
    \index{#2@\protect\texttt{#2} (Lua)}%
    \def\temp{#1}
    \ifx\temp\pgfutil@empty\else
      \index{#1@\protect\texttt{#1}!#2@\protect\texttt{#2} (Lua)}%
    \fi
  \pgfmanualbody
}{\endpgfmanualentry}

\newenvironment{luatable}[3]{
  \medskip
  \luageneric{#1}{#2}{ (declared in \texttt{#3})}{\textbf{Lua table}}
}{\endluageneric}

\newenvironment{luafield}[1]{
  \pgfmanualentry
    \pgfmanualentryheadline{Field \texttt{\declare{#1}}}
  \pgfmanualbody
}{\endpgfmanualentry}


\newenvironment{lualibrary}[1]{
  \pgfmanualentry
  \pgfmanualentryheadline{%
    \pgfmanualpdflabel{#1}{}%
    \textbf{Graph Drawing Library} \texttt{\declare{#1}}%
  }
    \index{#1@\protect\texttt{#1} graph drawing library}%
    \index{Libraries!#1@\protect\texttt{#1}}%
    \index{Graph drawing libraries!#1@\protect\texttt{#1}}%
    \vskip.25em
    {\ttfamily\char`\\usegdlibrary\char`\{\declare{#1}\char`\}\space\space \char`\%\space\space  \LaTeX\space and plain \TeX}\\
    {\ttfamily\char`\\usegdlibrary[\declare{#1}]\space \char`\%\space\space Con\TeX t}\smallskip\par
    \pgfmanualbody
}{\endpgfmanualentry}

\newenvironment{luadeclare}[4]{
  \pgfmanualentry
  \def\manual@temp@default{#3}%
  \def\manual@temp@initial{#4}%
  \def\manual@temp@{#3#4}%
  \pgfmanualentryheadline{%
    \pgfmanualpdflabel{#1}{}%
    {\ttfamily/graph
      drawing/\declare{#1}\opt{=}}\opt{#2}\hfill%
    \ifx\manual@temp@\pgfutil@empty\else%
    (\ifx\manual@temp@default\pgfutil@empty\else%
    default {\ttfamily #3}\ifx\manual@temp@initial\pgfutil@empty\else, \fi%
    \fi%
    \ifx\manual@temp@initial\pgfutil@empty\else%
    initially {\ttfamily #4}%
    \fi%
    )\fi%
  }%
  \index{#1@\protect\texttt{#1} key}%
  \pgfmanualbody
  \gdef\myname{#1}%
%  \keyalias{tikz}
%  \keyalias{tikz/graphs}
}{\endpgfmanualentry}

\newenvironment{luadeclarestyle}[4]{
  \pgfmanualentry
  \def\manual@temp@para{#2}%
  \def\manual@temp@default{#3}%
  \def\manual@temp@initial{#4}%
  \def\manual@temp@{#3#4}%
  \pgfmanualentryheadline{%
    \pgfmanualpdflabel{#1}{}%
    {\ttfamily/graph drawing/\declare{#1}}\ifx\manual@temp@para\pgfutil@empty\else\opt{\texttt=}\opt{#2}\fi\hfill%
    (style\ifx\manual@temp@\pgfutil@empty\else, %
    \ifx\manual@temp@default\pgfutil@empty\else%
    default {\ttfamily #3}\ifx\manual@temp@initial\pgfutil@empty\else, \fi%
    \fi%
    \ifx\manual@temp@initial\pgfutil@empty\else%
    initially {\ttfamily #4}%
    \fi%
    \fi)%
  }%
  \index{#1@\protect\texttt{#1} key}%
  \pgfmanualbody%
  \gdef\myname{#1}%
%  \keyalias{tikz}
%  \keyalias{tikz/graphs}
}{\endpgfmanualentry}

\newenvironment{luanamespace}[2]{
  \luageneric{#1}{#2}{}{\textbf{Lua namespace}}
}{\endluageneric}

\newenvironment{luafiledescription}[1]{}{}

\newenvironment{luacommand}[4]{
  \hypertarget{pgf/lua/#1}{\luageneric{#2}{#3}{\texttt{(#4)}}{\texttt{function}}}
}{\endluageneric}

\newenvironment{luaparameters}{\par\emph{Parameters:}%
  \parametercount=0\relax%
  \let\item=\parameteritem%
  \let\list=\restorelist%
}
{\par
}

\newenvironment{luareturns}{\par\emph{Returns:}%
  \parametercount=0\relax%
  \let\item=\parameteritem%
  \let\list=\restorelist%
}
{\par
}

\newcount\parametercount

\newenvironment{parameterdescription}{\unskip%
  \parametercount=0\relax%
  \let\item=\parameteritem%
  \let\list=\restorelist%
}
{\par
}
\let\saveditemcommand=\item
\let\savedlistcommand=\list
\def\denselist#1#2{\savedlistcommand{#1}{#2}\parskip0pt\itemsep0pt}
\def\restorelist{\let\item=\saveditemcommand\denselist}
\def\parameteritem{\pgfutil@ifnextchar[\parameteritem@{}}%}
\def\parameteritem@[#1]{\advance\parametercount by1\relax\hskip0.15em plus 1em\emph{\the\parametercount.}\kern1ex\def\test{#1}\ifx\test\pgfutil@empty\else#1\kern.5em\fi}

\newenvironment{commandlist}[1]{%
  \begin{pgfmanualentry}
  \foreach \xx in {#1} {%
    \expandafter\extractcommand\xx\@@
  }%
  \pgfmanualbody
}{%
  \end{pgfmanualentry}
}%

% \begin{internallist}[register]{\pgf@xa}
% \end{internallist}
%
% \begin{internallist}[register]{\pgf@xa,\pgf@xb}
% \end{internallist}
\newenvironment{internallist}[2][register]{%
  \begin{pgfmanualentry}
  \foreach \xx in {#2} {%
    \expandafter\extractinternalcommand\expandafter{\xx}{#1}%
  }%
  \pgfmanualbody
}{%
  \end{pgfmanualentry}
}%
\def\extractinternalcommand#1#2{%
  \removeats{#1}%
  \pgfmanualentryheadline{%
    \pgfmanualpdflabel{\textbackslash\strippedat}{}%
    Internal #2 \declare{\texttt{\string#1}}}%
  \index{Internals!\strippedat @\protect\myprintocmmand{\strippedat}}%
  \index{\strippedat @\protect\myprintocmmand{\strippedat}}%
}

%% MW: START MATH MACROS
\def\mvar#1{{\ifmmode\textrm{\textit{#1}}\else\rmfamily\textit{#1}\fi}}

\makeatletter

\def\extractmathfunctionname#1{\extractmathfunctionname@#1(,)\tmpa\tmpb}
\def\extractmathfunctionname@#1(#2)#3\tmpb{\def\mathname{#1}}

\makeatother

\newenvironment{math-function}[1]{
  \def\mathdefaultname{#1}
  \extractmathfunctionname{#1}
  \edef\mathurl{{math:\mathname}}\expandafter\hypertarget\expandafter{\mathurl}{}%
  \begin{pgfmanualentry}
    \pgfmanualentryheadline{\texttt{#1}}%
    \index{\mathname @\protect\texttt{\mathname} math function}%
    \index{Math functions!\mathname @\protect\texttt{\mathname}}%
    \pgfmanualbody
}
{
  \end{pgfmanualentry}
}

\def\pgfmanualemptytext{}
\def\pgfmanualvbarvbar{\char`\|\char`\|}

\newenvironment{math-operator}[4][]{%
  \begin{pgfmanualentry}
  \csname math#3operator\endcsname{#2}{#4}
  \def\mathtest{#4}%
  \ifx\mathtest\pgfmanualemptytext%
    \def\mathtype{(#3 operator)}
  \else%
    \def\mathtype{(#3 operator; uses the \texttt{#4} function)}
  \fi%
  \pgfmanualentryheadline{\mathexample\hfill\mathtype}%
  \def\mathtest{#1}%
  \ifx\mathtest\pgfmanualemptytext%
    \index{#2@\protect\texttt{#2} #3 math operator}%
    \index{Math operators!#2@\protect\texttt{#2}}%
  \fi%
  \pgfmanualbody
}
{\end{pgfmanualentry}}

\newenvironment{math-operators}[5][]{%
  \begin{pgfmanualentry}
  \csname math#4operator\endcsname{#2}{#3}
  \def\mathtest{#5}%
  \ifx\mathtest\pgfmanualemptytext%
    \def\mathtype{(#4 operators)}
  \else%
    \def\mathtype{(#4 operators; use the \texttt{#5} function)}
  \fi%
  \pgfmanualentryheadline{\mathexample\hfill\mathtype}%
  \def\mathtest{#1}%
  \ifx\mathtest\pgfmanualemptytext%
    \index{#2#3@\protect\texttt{#2\protect\ #3} #4 math operators}%
    \index{Math operators!#2#3@\protect\texttt{#2\protect\ #3}}%
  \fi%
  \pgfmanualbody
}
{\end{pgfmanualentry}}

\def\mathinfixoperator#1#2{%
  \def\mathoperator{\texttt{#1}}%
  \def\mathexample{\mvar{x}\space\texttt{#1}\space\mvar{y}}%
}

\def\mathprefixoperator#1#2{%
  \def\mathoperator{\texttt{#1}}%
  \def\mathexample{\texttt{#1}\mvar{x}}%
}

\def\mathpostfixoperator#1#2{%
  \def\mathoperator{\texttt{#1}}
  \def\mathexample{\mvar{x}\texttt{#1}}%
}

\def\mathgroupoperator#1#2{%
  \def\mathoperator{\texttt{#1\ #2}}%
  \def\mathexample{\texttt{#1}\mvar{x}\texttt{#2}}%
}

\expandafter\let\csname matharray accessoperator\endcsname=\mathgroupoperator
\expandafter\let\csname matharrayoperator\endcsname=\mathgroupoperator

\def\mathconditionaloperator#1#2{%
  \def\mathoperator{#1\space#2}
  \def\mathexample{\mvar{x}\ \texttt{#1}\ \mvar{y}\ {\texttt{#2}}\ \mvar{z}}
}

\newcommand\mathcommand[1][\mathdefaultname]{%
  \expandafter\makemathcommand#1(\empty)\stop%
  \expandafter\extractcommand\mathcommandname\@@%
  \medskip
}
\makeatletter

\def\makemathcommand#1(#2)#3\stop{%
  \expandafter\def\expandafter\mathcommandname\expandafter{\csname pgfmath#1\endcsname}%
  \ifx#2\empty%
  \else%
    \@makemathcommand#2,\stop,
  \fi}
\def\@makemathcommand#1,{%
  \ifx#1\stop%
  \else%
    \expandafter\def\expandafter\mathcommandname\expandafter{\mathcommandname{\ttfamily\char`\{#1\char`\}}}%
    \expandafter\@makemathcommand%
  \fi}
\makeatother

\def\calcname{\textsc{calc}}

\newenvironment{math-keyword}[1]{
  \extracttikzmathkeyword#1@
  \begin{pgfmanualentry}
    \pgfmanualentryheadline{\texttt{\color{red}\mathname}\mathrest}%
    \index{\mathname @\protect\texttt{\mathname} tikz math function}%
    \index{TikZ math functions!\mathname @\protect\texttt{\mathname}}%
    \pgfmanualbody
}
{
  \end{pgfmanualentry}
}

\def\extracttikzmathkeyword#1#2@{%
  \def\mathname{#1}%
  \def\mathrest{#2}%
}

%% MW: END MATH MACROS


\def\extractcommand#1#2\@@{%
  \removeats{#1}%
  \pgfmanualentryheadline{%
    \pgfmanualpdflabel{\textbackslash\strippedat}{}%
    \declare{\texttt{\string#1}}#2%
  }%
  \index{\strippedat @\protect\myprintocmmand{\strippedat}}
}

\def\luaextractcommand#1#2\relax{%
  \declare{\texttt{\string#1}}#2\par%
%  \removeats{#1}%
 % \index{\strippedat @\protect\myprintocmmand{\strippedat}}
 % \pgfmanualpdflabel{\textbackslash\strippedat}{}%
}


% \begin{environment}{{name}\marg{arguments}}
\renewenvironment{environment}[1]{
  \begin{pgfmanualentry}
    \extractenvironement#1\@@
    \pgfmanualbody
}
{
  \end{pgfmanualentry}
}

\def\extractenvironement#1#2\@@{%
  \pgfmanualentryheadline{%
    \pgfmanualpdflabel{#1}{}%
    {\ttfamily\char`\\begin\char`\{\declare{#1}\char`\}}#2%
  }%
  \pgfmanualentryheadline{{\ttfamily\ \ }\meta{environment contents}}%
  \pgfmanualentryheadline{{\ttfamily\char`\\end\char`\{\declare{#1}\char`\}}}%
  \index{#1@\protect\texttt{#1} environment}%
  \index{Environments!#1@\protect\texttt{#1}}
}


\newenvironment{plainenvironment}[1]{
  \begin{pgfmanualentry}
    \extractplainenvironement#1\@@
    \pgfmanualbody
}
{
  \end{pgfmanualentry}
}

\def\extractplainenvironement#1#2\@@{%
  \pgfmanualentryheadline{{\ttfamily\declare{\char`\\#1}}#2}%
  \pgfmanualentryheadline{{\ttfamily\ \ }\meta{environment contents}}%
  \pgfmanualentryheadline{{\ttfamily\declare{\char`\\end#1}}}%
  \index{#1@\protect\texttt{#1} environment}%
  \index{Environments!#1@\protect\texttt{#1}}%
}


\newenvironment{contextenvironment}[1]{
  \begin{pgfmanualentry}
    \extractcontextenvironement#1\@@
    \pgfmanualbody
}
{
  \end{pgfmanualentry}
}

\def\extractcontextenvironement#1#2\@@{%
  \pgfmanualentryheadline{{\ttfamily\declare{\char`\\start#1}}#2}%
  \pgfmanualentryheadline{{\ttfamily\ \ }\meta{environment contents}}%
  \pgfmanualentryheadline{{\ttfamily\declare{\char`\\stop#1}}}%
  \index{#1@\protect\texttt{#1} environment}%
  \index{Environments!#1@\protect\texttt{#1}}}


\newenvironment{shape}[1]{
  \begin{pgfmanualentry}
    \pgfmanualentryheadline{%
      \pgfmanualpdflabel{#1}{}%
      \textbf{Shape} {\ttfamily\declare{#1}}%
    }%
    \index{#1@\protect\texttt{#1} shape}%
    \index{Shapes!#1@\protect\texttt{#1}}
    \pgfmanualbody
}
{
  \end{pgfmanualentry}
}

\newenvironment{pictype}[2]{
  \begin{pgfmanualentry}
    \pgfmanualentryheadline{%
      \pgfmanualpdflabel{#1}{}%
      \textbf{Pic type} {\ttfamily\declare{#1}#2}%
    }%
    \index{#1@\protect\texttt{#1} pic type}%
    \index{Pic Types!#1@\protect\texttt{#1}}
    \pgfmanualbody
}
{
  \end{pgfmanualentry}
}

\newenvironment{shading}[1]{
  \begin{pgfmanualentry}
    \pgfmanualentryheadline{%
      \pgfmanualpdflabel{#1}{}%
      \textbf{Shading} {\ttfamily\declare{#1}}}%
    \index{#1@\protect\texttt{#1} shading}%
    \index{Shadings!#1@\protect\texttt{#1}}
    \pgfmanualbody
}
{
  \end{pgfmanualentry}
}


\newenvironment{graph}[1]{
  \begin{pgfmanualentry}
    \pgfmanualentryheadline{%
      \pgfmanualpdflabel{#1}{}%
      \textbf{Graph} {\ttfamily\declare{#1}}}%
    \index{#1@\protect\texttt{#1} graph}%
    \index{Graphs!#1@\protect\texttt{#1}}
    \pgfmanualbody
}
{
  \end{pgfmanualentry}
}

\newenvironment{gdalgorithm}[2]{
  \begin{pgfmanualentry}
    \pgfmanualentryheadline{%
      \pgfmanualpdflabel{#1}{}%
      \textbf{Layout} {\ttfamily/graph drawing/\declare{#1}\opt{=}}\opt{\meta{options}}}%
    \index{#1@\protect\texttt{#1} layout}%
    \index{Layouts!#1@\protect\texttt{#1}}%
    \foreach \algo in {#2}
    {\edef\marshal{\noexpand\index{#2@\noexpand\protect\noexpand\texttt{#2} algorithm}}\marshal}%
    \index{Graph drawing layouts!#1@\protect\texttt{#1}}
    \item{\small alias {\ttfamily/tikz/#1}}\par
    \item{\small alias {\ttfamily/tikz/graphs/#1}}\par
    \item{\small Employs {\ttfamily algorithm=#2}}\par
    \pgfmanualbody
}
{
  \end{pgfmanualentry}
}

\newenvironment{dataformat}[1]{
  \begin{pgfmanualentry}
    \pgfmanualentryheadline{%
      \pgfmanualpdflabel{#1}{}%
      \textbf{Format} {\ttfamily\declare{#1}}}%
    \index{#1@\protect\texttt{#1} format}%
    \index{Formats!#1@\protect\texttt{#1}}
    \pgfmanualbody
}
{
  \end{pgfmanualentry}
}

\newenvironment{stylesheet}[1]{
  \begin{pgfmanualentry}
    \pgfmanualentryheadline{%
      \pgfmanualpdflabel{#1}{}%
      \textbf{Style sheet} {\ttfamily\declare{#1}}}%
    \index{#1@\protect\texttt{#1} style sheet}%
    \index{Style sheets!#1@\protect\texttt{#1}}
    \pgfmanualbody
}
{
  \end{pgfmanualentry}
}

\newenvironment{handler}[1]{
  \begin{pgfmanualentry}
    \extracthandler#1\@nil%
    \pgfmanualbody
}
{
  \end{pgfmanualentry}
}

\def\gobble#1{}
\def\extracthandler#1#2\@nil{%
  \pgfmanualentryheadline{%
    \pgfmanualpdflabel{/handlers/#1}{}%
    \textbf{Key handler} \meta{key}{\ttfamily/\declare{#1}}#2}%
  \index{\gobble#1@\protect\texttt{#1} handler}%
  \index{Key handlers!#1@\protect\texttt{#1}}
}


\makeatletter


\newenvironment{stylekey}[1]{
  \begin{pgfmanualentry}
    \def\extrakeytext{style, }
    \extractkey#1\@nil%
    \pgfmanualbody
}
{
  \end{pgfmanualentry}
}

\def\choicesep{$\vert$}%
\def\choicearg#1{\texttt{#1}}

\newif\iffirstchoice

% \mchoice{choice1,choice2,choice3}
\newcommand\mchoice[1]{%
  \begingroup
  \firstchoicetrue
  \foreach \mchoice@ in {#1} {%
    \iffirstchoice
      \global\firstchoicefalse
    \else
      \choicesep
    \fi
    \choicearg{\mchoice@}%
  }%
  \endgroup
}%

% \begin{key}{/path/x=value}
% \begin{key}{/path/x=value (initially XXX)}
% \begin{key}{/path/x=value (default XXX)}
\newenvironment{key}[1]{
  \begin{pgfmanualentry}
    \def\extrakeytext{}
    %\def\altpath{\emph{\color{gray}or}}%
    \extractkey#1\@nil%
    \pgfmanualbody
}
{
  \end{pgfmanualentry}
}

% \insertpathifneeded{a key}{/pgf} -> assign mykey={/pgf/a key}
% \insertpathifneeded{/tikz/a key}{/pgf} -> assign mykey={/tikz/a key}
%
% #1: the key
% #2: a default path (or empty)
\def\insertpathifneeded#1#2{%
  \def\insertpathifneeded@@{#2}%
  \ifx\insertpathifneeded@@\empty
    \def\mykey{#1}%
  \else
    \insertpathifneeded@#2\@nil
    \ifpgfutil@in@
      \def\mykey{#2/#1}%
    \else
      \def\mykey{#1}%
    \fi
  \fi
}%
\def\insertpathifneeded@#1#2\@nil{%
  \def\insertpathifneeded@@{#1}%
  \def\insertpathifneeded@@@{/}%
  \ifx\insertpathifneeded@@\insertpathifneeded@@@
    \pgfutil@in@true
  \else
    \pgfutil@in@false
  \fi
}%

% \begin{keylist}[default path]
%   {/path/option 1=value,/path/option 2=value2}
% \end{keylist}
\newenvironment{keylist}[2][]{%
  \begin{pgfmanualentry}
    \def\extrakeytext{}%
  \foreach \xx in {#2} {%
    \expandafter\insertpathifneeded\expandafter{\xx}{#1}%
    \expandafter\extractkey\mykey\@nil%
  }%
  \pgfmanualbody
}{%
  \end{pgfmanualentry}
}%

\def\extractkey#1\@nil{%
  \pgfutil@in@={#1}%
  \ifpgfutil@in@%
    \extractkeyequal#1\@nil
  \else%
    \pgfutil@in@{(initial}{#1}%
    \ifpgfutil@in@%
      \extractequalinitial#1\@nil%
    \else
      \pgfmanualentryheadline{%
      \def\mykey{#1}%
      \def\mypath{}%
      \gdef\myname{}%
      \firsttimetrue%
      \pgfmanualdecomposecount=0\relax%
      \decompose#1/\nil%
        {\ttfamily\declare{#1}}\hfill(\extrakeytext no value)}%
    \fi
  \fi%
}

\def\extractkeyequal#1=#2\@nil{%
  \pgfutil@in@{(default}{#2}%
  \ifpgfutil@in@%
    \extractdefault{#1}#2\@nil%
  \else%
    \pgfutil@in@{(initial}{#2}%
    \ifpgfutil@in@%
      \extractinitial{#1}#2\@nil%
    \else
      \pgfmanualentryheadline{%
        \def\mykey{#1}%
        \def\mypath{}%
        \gdef\myname{}%
        \firsttimetrue%
        \pgfmanualdecomposecount=0\relax%
        \decompose#1/\nil%
        {\ttfamily\declare{#1}=}#2\hfill(\extrakeytext no default)}%
    \fi%
  \fi%
}

\def\extractdefault#1#2(default #3)\@nil{%
  \pgfmanualentryheadline{%
    \def\mykey{#1}%
    \def\mypath{}%
    \gdef\myname{}%
    \firsttimetrue%
    \pgfmanualdecomposecount=0\relax%
    \decompose#1/\nil%
    {\ttfamily\declare{#1}\opt{=}}\opt{#2}\hfill (\extrakeytext default {\ttfamily#3})}%
}

\def\extractinitial#1#2(initially #3)\@nil{%
  \pgfmanualentryheadline{%
    \def\mykey{#1}%
    \def\mypath{}%
    \gdef\myname{}%
    \firsttimetrue%
    \pgfmanualdecomposecount=0\relax%
    \decompose#1/\nil%
    {\ttfamily\declare{#1}=}#2\hfill (\extrakeytext no default, initially {\ttfamily#3})}%
}

\def\extractequalinitial#1 (initially #2)\@nil{%
  \pgfmanualentryheadline{%
    \def\mykey{#1}%
    \def\mypath{}%
    \gdef\myname{}%
    \firsttimetrue%
    \pgfmanualdecomposecount=0\relax%
    \decompose#1/\nil%
    {\ttfamily\declare{#1}}\hfill (\extrakeytext initially {\ttfamily#2})}%
}

% Introduces a key alias '/#1/<name of current key>'
% to be used inside of \begin{key} ... \end{key}
\def\keyalias#1{\vspace{-3pt}\item{\small alias {\ttfamily/#1/\myname}}\vspace{-2pt}\par
  \pgfmanualpdflabel{/#1/\myname}{}%
}

\newif\iffirsttime
\newcount\pgfmanualdecomposecount

\makeatother

\def\decompose/#1/#2\nil{%
  \def\test{#2}%
  \ifx\test\empty%
    % aha.
    \index{#1@\protect\texttt{#1} key}%
    \index{\mypath#1@\protect\texttt{#1}}%
    \gdef\myname{#1}%
    \pgfmanualpdflabel{#1}{}
  \else%
    \advance\pgfmanualdecomposecount by1\relax%
    \ifnum\pgfmanualdecomposecount>2\relax%
      \decomposetoodeep#1/#2\nil%
    \else%
      \iffirsttime%
        \begingroup%
          % also make a pdf link anchor with full key path.
          \def\hyperlabelwithoutslash##1/\nil{%
            \pgfmanualpdflabel{##1}{}%
          }%
          \hyperlabelwithoutslash/#1/#2\nil%
        \endgroup%
        \def\mypath{#1@\protect\texttt{/#1/}!}%
        \firsttimefalse%
      \else%
        \expandafter\def\expandafter\mypath\expandafter{\mypath#1@\protect\texttt{#1/}!}%
      \fi%
      \def\firsttime{}%
      \decompose/#2\nil%
    \fi%
  \fi%
}

\def\decomposetoodeep#1/#2/\nil{%
  % avoid too-deep nesting in index
  \index{#1/#2@\protect\texttt{#1/#2} key}%
  \index{\mypath#1/#2@\protect\texttt{#1/#2}}%
  \decomposefindlast/#1/#2/\nil%
}
\makeatletter
\def\decomposefindlast/#1/#2\nil{%
  \def\test{#2}%
  \ifx\test\pgfutil@empty%
    \gdef\myname{#1}%
  \else%
    \decomposefindlast/#2\nil%
  \fi%
}
\makeatother
\def\indexkey#1{%
  \def\mypath{}%
  \decompose#1/\nil%
}

\newenvironment{predefinedmethod}[1]{
  \begin{pgfmanualentry}
    \extractpredefinedmethod#1\@nil
    \pgfmanualbody
}
{
  \end{pgfmanualentry}
}
\def\extractpredefinedmethod#1(#2)\@nil{%
  \pgfmanualentryheadline{%
    \pgfmanualpdflabel{#1}{}%
    Method \declare{\ttfamily #1}\texttt(#2\texttt) \hfill(predefined for all classes)}
  \index{#1@\protect\texttt{#1} method}%
  \index{Methods!#1@\protect\texttt{#1}}
}


\newenvironment{ooclass}[1]{
  \begin{pgfmanualentry}
    \def\currentclass{#1}
    \pgfmanualentryheadline{%
      \pgfmanualpdflabel{#1}{}%
      \textbf{Class} \declare{\texttt{#1}}}
    \index{#1@\protect\texttt{#1} class}%
    \index{Class #1@Class \protect\texttt{#1}}%
    \index{Classes!#1@\protect\texttt{#1}}
    \pgfmanualbody
}
{
  \end{pgfmanualentry}
}

\newenvironment{method}[1]{
  \begin{pgfmanualentry}
    \extractmethod#1\@nil
    \pgfmanualbody
}
{
  \end{pgfmanualentry}
}
\def\extractmethod#1(#2)\@nil{%
  \def\test{#1}
  \ifx\test\currentclass
    \pgfmanualentryheadline{%
      \pgfmanualpdflabel{#1}{}%
      Constructor \declare{\ttfamily #1}\texttt(#2\texttt)}
  \else
    \pgfmanualentryheadline{%
      \pgfmanualpdflabel{#1}{}%
      Method \declare{\ttfamily #1}\texttt(#2\texttt)}
  \fi
  \index{#1@\protect\texttt{#1} method}%
  \index{Methods!#1@\protect\texttt{#1}}
  \index{Class \currentclass!#1@\protect\texttt{#1}}%
}

\newenvironment{attribute}[1]{
  \begin{pgfmanualentry}
    \extractattribute#1\@nil
    \pgfmanualbody
}
{
  \end{pgfmanualentry}
}
\def\extractattribute#1=#2;\@nil{%
  \def\test{#2}%
  \ifx\test\@empty
    \pgfmanualentryheadline{%
      \pgfmanualpdflabel{#1}{}%
      Private attribute \declare{\ttfamily #1} \hfill (initially empty)}
  \else
    \pgfmanualentryheadline{%
      \pgfmanualpdflabel{#1}{}%
      Private attribute \declare{\ttfamily #1} \hfill (initially {\ttfamily #2})}
  \fi
  \index{#1@\protect\texttt{#1} attribute}%
  \index{Attributes!#1@\protect\texttt{#1}}
  \index{Class \currentclass!#1@\protect\texttt{#1}}%
}



\newenvironment{predefinednode}[1]{
  \begin{pgfmanualentry}
    \pgfmanualentryheadline{%
      \pgfmanualpdflabel{#1}{}%
      \textbf{Predefined node} {\ttfamily\declare{#1}}}%
    \index{#1@\protect\texttt{#1} node}%
    \index{Predefined node!#1@\protect\texttt{#1}}
    \pgfmanualbody
}
{
  \end{pgfmanualentry}
}

\newenvironment{coordinatesystem}[1]{
  \begin{pgfmanualentry}
    \pgfmanualentryheadline{%
      \pgfmanualpdflabel{#1}{}%
      \textbf{Coordinate system} {\ttfamily\declare{#1}}}%
    \index{#1@\protect\texttt{#1} coordinate system}%
    \index{Coordinate systems!#1@\protect\texttt{#1}}
    \pgfmanualbody
}
{
  \end{pgfmanualentry}
}

\newenvironment{snake}[1]{
  \begin{pgfmanualentry}
    \pgfmanualentryheadline{\textbf{Snake} {\ttfamily\declare{#1}}}%
    \index{#1@\protect\texttt{#1} snake}%
    \index{Snakes!#1@\protect\texttt{#1}}
    \pgfmanualbody
}
{
  \end{pgfmanualentry}
}

\newenvironment{decoration}[1]{
  \begin{pgfmanualentry}
    \pgfmanualentryheadline{\textbf{Decoration} {\ttfamily\declare{#1}}}%
    \index{#1@\protect\texttt{#1} decoration}%
    \index{Decorations!#1@\protect\texttt{#1}}
    \pgfmanualbody
}
{
  \end{pgfmanualentry}
}


\def\pgfmanualbar{\char`\|}
\makeatletter
\newenvironment{pathoperation}[3][]{
  \begin{pgfmanualentry}
    \def\pgfmanualtest{#1}%
    \pgfmanualentryheadline{%
      \ifx\pgfmanualtest\@empty%
        \pgfmanualpdflabel{#2}{}%
      \fi%
      \textcolor{gray}{{\ttfamily\char`\\path}\
        \ \dots}
      \declare{\texttt{\noligs{#2}}}#3\ \textcolor{gray}{\dots\texttt{;}}}%
    \ifx\pgfmanualtest\@empty%
      \index{#2@\protect\texttt{#2} path operation}%
      \index{Path operations!#2@\protect\texttt{#2}}%
    \fi%
    \pgfmanualbody
}
{
  \end{pgfmanualentry}
}
\newenvironment{datavisualizationoperation}[3][]{
  \begin{pgfmanualentry}
    \def\pgfmanualtest{#1}%
    \pgfmanualentryheadline{%
      \ifx\pgfmanualtest\@empty%
        \pgfmanualpdflabel{#2}{}%
      \fi%
      \textcolor{gray}{{\ttfamily\char`\\datavisualization}\
        \ \dots}
      \declare{\texttt{\noligs{#2}}}#3\ \textcolor{gray}{\dots\texttt{;}}}%
    \ifx\pgfmanualtest\@empty%
      \index{#2@\protect\texttt{#2} (data visualization)}%
      \index{Data visualization!#2@\protect\texttt{#2}}%
    \fi%
    \pgfmanualbody
}
{
  \end{pgfmanualentry}
}
\makeatother

\def\doublebs{\texttt{\char`\\\char`\\}}


\newenvironment{package}[1]{
  \begin{pgfmanualentry}
    \pgfmanualentryheadline{%
      \pgfmanualpdflabel{#1}{}%
      {\ttfamily\char`\\usepackage\char`\{\declare{#1}\char`\}\space\space \char`\%\space\space  \LaTeX}}
    \index{#1@\protect\texttt{#1} package}%
    \index{Packages and files!#1@\protect\texttt{#1}}%
    \pgfmanualentryheadline{{\ttfamily\char`\\input \declare{#1}.tex\space\space\space \char`\%\space\space  plain \TeX}}
    \pgfmanualentryheadline{{\ttfamily\char`\\usemodule[\declare{#1}]\space\space \char`\%\space\space  Con\TeX t}}
    \pgfmanualbody
}
{
  \end{pgfmanualentry}
}


\newenvironment{pgfmodule}[1]{
  \begin{pgfmanualentry}
    \pgfmanualentryheadline{%
      \pgfmanualpdflabel{#1}{}%
      {\ttfamily\char`\\usepgfmodule\char`\{\declare{#1}\char`\}\space\space\space
        \char`\%\space\space  \LaTeX\space and plain \TeX\space and pure pgf}}
    \index{#1@\protect\texttt{#1} module}%
    \index{Modules!#1@\protect\texttt{#1}}%
    \pgfmanualentryheadline{{\ttfamily\char`\\usepgfmodule[\declare{#1}]\space\space \char`\%\space\space  Con\TeX t\space and pure pgf}}
    \pgfmanualbody
}
{
  \end{pgfmanualentry}
}

\newenvironment{pgflibrary}[1]{
  \begin{pgfmanualentry}
    \pgfmanualentryheadline{%
      \pgfmanualpdflabel{#1}{}%
      \textbf{\tikzname\ Library} \texttt{\declare{#1}}}
    \index{#1@\protect\texttt{#1} library}%
    \index{Libraries!#1@\protect\texttt{#1}}%
    \vskip.25em%
    {{\ttfamily\char`\\usepgflibrary\char`\{\declare{#1}\char`\}\space\space\space
        \char`\%\space\space  \LaTeX\space and plain \TeX\space and pure pgf}}\\
    {{\ttfamily\char`\\usepgflibrary[\declare{#1}]\space\space \char`\%\space\space  Con\TeX t\space and pure pgf}}\\
    {{\ttfamily\char`\\usetikzlibrary\char`\{\declare{#1}\char`\}\space\space
        \char`\%\space\space  \LaTeX\space and plain \TeX\space when using \tikzname}}\\
    {{\ttfamily\char`\\usetikzlibrary[\declare{#1}]\space
        \char`\%\space\space  Con\TeX t\space when using \tikzname}}\\[.5em]
    \pgfmanualbody
}
{
  \end{pgfmanualentry}
}

\newenvironment{purepgflibrary}[1]{
  \begin{pgfmanualentry}
    \pgfmanualentryheadline{%
      \pgfmanualpdflabel{#1}{}%
      \textbf{{\small PGF} Library} \texttt{\declare{#1}}}
    \index{#1@\protect\texttt{#1} library}%
    \index{Libraries!#1@\protect\texttt{#1}}%
    \vskip.25em%
    {{\ttfamily\char`\\usepgflibrary\char`\{\declare{#1}\char`\}\space\space\space
        \char`\%\space\space  \LaTeX\space and plain \TeX}}\\
    {{\ttfamily\char`\\usepgflibrary[\declare{#1}]\space\space \char`\%\space\space  Con\TeX t}}\\[.5em]
    \pgfmanualbody
}
{
  \end{pgfmanualentry}
}

\newenvironment{tikzlibrary}[1]{
  \begin{pgfmanualentry}
    \pgfmanualentryheadline{%
      \pgfmanualpdflabel{#1}{}%
      \textbf{\tikzname\ Library} \texttt{\declare{#1}}}
    \index{#1@\protect\texttt{#1} library}%
    \index{Libraries!#1@\protect\texttt{#1}}%
    \vskip.25em%
    {{\ttfamily\char`\\usetikzlibrary\char`\{\declare{#1}\char`\}\space\space \char`\%\space\space  \LaTeX\space and plain \TeX}}\\
    {{\ttfamily\char`\\usetikzlibrary[\declare{#1}]\space \char`\%\space\space Con\TeX t}}\\[.5em]
    \pgfmanualbody
}
{
  \end{pgfmanualentry}
}



\newenvironment{filedescription}[1]{
  \begin{pgfmanualentry}
    \pgfmanualentryheadline{File {\ttfamily\declare{#1}}}%
    \index{#1@\protect\texttt{#1} file}%
    \index{Packages and files!#1@\protect\texttt{#1}}%
    \pgfmanualbody
}
{
  \end{pgfmanualentry}
}


\newenvironment{packageoption}[1]{
  \begin{pgfmanualentry}
    \pgfmanualentryheadline{{\ttfamily\char`\\usepackage[\declare{#1}]\char`\{pgf\char`\}}}
    \index{#1@\protect\texttt{#1} package option}%
    \index{Package options for \textsc{pgf}!#1@\protect\texttt{#1}}%
    \pgfmanualbody
}
{
  \end{pgfmanualentry}
}



\newcommand\opt[1]{{\color{black!50!green}#1}}
\newcommand\ooarg[1]{{\ttfamily[}\meta{#1}{\ttfamily]}}

\def\opt{\afterassignment\pgfmanualopt\let\next=}
\def\pgfmanualopt{\ifx\next\bgroup\bgroup\color{black!50!green}\else{\color{black!50!green}\next}\fi}



\def\beamer{\textsc{beamer}}
\def\pdf{\textsc{pdf}}
\def\eps{\texttt{eps}}
\def\pgfname{\textsc{pgf}}
\def\tikzname{Ti\emph{k}Z}
\def\pstricks{\textsc{pstricks}}
\def\prosper{\textsc{prosper}}
\def\seminar{\textsc{seminar}}
\def\texpower{\textsc{texpower}}
\def\foils{\textsc{foils}}

{
  \makeatletter
  \global\let\myempty=\@empty
  \global\let\mygobble=\@gobble
  \catcode`\@=12
  \gdef\getridofats#1@#2\relax{%
    \def\getridtest{#2}%
    \ifx\getridtest\myempty%
      \expandafter\def\expandafter\strippedat\expandafter{\strippedat#1}
    \else%
      \expandafter\def\expandafter\strippedat\expandafter{\strippedat#1\protect\printanat}
      \getridofats#2\relax%
    \fi%
  }

  \gdef\removeats#1{%
    \let\strippedat\myempty%
    \edef\strippedtext{\stripcommand#1}%
    \expandafter\getridofats\strippedtext @\relax%
  }

  \gdef\stripcommand#1{\expandafter\mygobble\string#1}
}

\def\printanat{\char`\@}

\def\declare{\afterassignment\pgfmanualdeclare\let\next=}
\def\pgfmanualdeclare{\ifx\next\bgroup\bgroup\color{red!75!black}\else{\color{red!75!black}\next}\fi}


\let\textoken=\command
\let\endtextoken=\endcommand

\def\myprintocmmand#1{\texttt{\char`\\#1}}

\def\example{\par\smallskip\noindent\textit{Example: }}
\def\themeauthor{\par\smallskip\noindent\textit{Theme author: }}


\def\indexoption#1{%
  \index{#1@\protect\texttt{#1} option}%
  \index{Graphic options and styles!#1@\protect\texttt{#1}}%
}

\def\itemcalendaroption#1{\item \declare{\texttt{#1}}%
  \index{#1@\protect\texttt{#1} date test}%
  \index{Date tests!#1@\protect\texttt{#1}}%
}



\def\class#1{\list{}{\leftmargin=2em\itemindent-\leftmargin\def\makelabel##1{\hss##1}}%
\extractclass#1@\par\topsep=0pt}
\def\endclass{\endlist}
\def\extractclass#1#2@{%
\item{{{\ttfamily\char`\\documentclass}#2{\ttfamily\char`\{\declare{#1}\char`\}}}}%
  \index{#1@\protect\texttt{#1} class}%
  \index{Classes!#1@\protect\texttt{#1}}}

\def\partname{Part}

\makeatletter
\def\index@prologue{\section*{Index}\addcontentsline{toc}{section}{Index}
  This index only contains automatically generated entries. A good
  index should also contain carefully selected keywords. This index is
  not a good index.
  \bigskip
}
\c@IndexColumns=2
  \def\theindex{\@restonecoltrue
    \columnseprule \z@  \columnsep 29\p@
    \twocolumn[\index@prologue]%
       \parindent -30pt
       \columnsep 15pt
       \parskip 0pt plus 1pt
       \leftskip 30pt
       \rightskip 0pt plus 2cm
       \small
       \def\@idxitem{\par}%
    \let\item\@idxitem \ignorespaces}
  \def\endtheindex{\onecolumn}
\def\noindexing{\let\index=\@gobble}


\newenvironment{arrowtipsimple}[1]{
  \begin{pgfmanualentry}
    \pgfmanualentryheadline{\textbf{Arrow Tip Kind} {\ttfamily#1}}
    \index{#1@\protect\texttt{#1} arrow tip}%
    \index{Arrow tips!#1@\protect\texttt{#1}}%
    \def\currentarrowtype{#1}
    \pgfmanualbody}
{
  \end{pgfmanualentry}
}

\newenvironment{arrowtip}[4]{
  \begin{pgfmanualentry}
    \pgfmanualentryheadline{\textbf{Arrow Tip Kind} {\ttfamily#1}}
    \index{#1@\protect\texttt{#1} arrow tip}%
    \index{Arrow tips!#1@\protect\texttt{#1}}%
    \pgfmanualbody
    \def\currentarrowtype{#1}
    \begin{minipage}[t]{10.25cm}
      #2
    \end{minipage}\hskip5mm\begin{minipage}[t]{4.75cm}
      \leavevmode\vskip-2em
    \tikz{
      \draw [black!50,line width=5mm,-{#1[#3,color=black]}] (-4,0) -- (0,0);
      \foreach \action in {#4}
      { \expandafter\processaction\action\relax }
    }
    \end{minipage}\par\smallskip
  }
{
  \end{pgfmanualentry}
}

\newenvironment{arrowcap}[5]{
  \begin{pgfmanualentry}
    \pgfmanualentryheadline{\textbf{Arrow Tip Kind} {\ttfamily#1}}
    \index{#1@\protect\texttt{#1} arrow tip}%
    \index{Arrow tips!#1@\protect\texttt{#1}}%
    \pgfmanualbody
    \def\currentarrowtype{#1}
    \begin{minipage}[t]{10.25cm}
      #2
    \end{minipage}\hskip5mm\begin{minipage}[t]{4.75cm}
      \leavevmode\vskip-2em
    \tikz{
      \path [tips, line width=10mm,-{#1[#3,color=black]}] (-4,0) -- (0,0);
      \draw [line width=10mm,black!50] (-3,0) -- (#5,0);
      \foreach \action in {#4}
      { \expandafter\processaction\action\relax }
    }
    \end{minipage}\par\smallskip
  }
{
  \end{pgfmanualentry}
}

\def\processaction#1=#2\relax{
  \expandafter\let\expandafter\pgf@temp\csname manual@action@#1\endcsname
  \ifx\pgf@temp\relax\else
    \pgf@temp#2/0/\relax
  \fi
}
\def\manual@action@length#1/#2/#3\relax{%
  \draw [red,|<->|,semithick,xshift=#2] ([yshift=4pt]current bounding
  box.north -| -#1,0) coordinate (last length) -- node
  [above=-2pt] {|length|} ++(#1,0);
}
\def\manual@action@width#1/#2/#3\relax{%
  \draw [overlay, red,|<->|,semithick] (.5,-#1/2) -- node [below,sloped] {|width|} (.5,#1/2);
}
\def\manual@action@inset#1/#2/#3\relax{%
  \draw [red,|<->|,semithick,xshift=#2] ([yshift=-4pt]current bounding
  box.south -| last length) -- node [below] {|inset|} ++(#1,0);
}

\newenvironment{arrowexamples}
{\begin{tabbing}
    \hbox to \dimexpr\linewidth-5.5cm\relax{\emph{Appearance of the below at line width} \hfil} \=
     \hbox to 1.9cm{\emph{0.4pt}\hfil} \= \hbox to 2cm{\emph{0.8pt}\hfil} \= \emph{1.6pt} \\
  }
{\end{tabbing}\vskip-1em}

\newenvironment{arrowcapexamples}
{\begin{tabbing}
    \hbox to \dimexpr\linewidth-5.5cm\relax{\emph{Appearance of the below at line width} \hfil} \=
     \hbox to 1.9cm{\emph{1ex}\hfil} \= \hbox to 2cm{\emph{1em}\hfil} \\
  }
{\end{tabbing}\vskip-1em}

\def\arrowcapexample#1[#2]{\def\temp{#1}\ifx\temp\pgfutil@empty\arrowcapexample@\currentarrowtype[{#2}]\else\arrowcapexample@#1[{#2}]\fi}
\def\arrowcapexample@#1[#2]{%
  {\sfcode`\.1000\small\texttt{#1[#2]}} \>
  \kern-.5ex\tikz [baseline,>={#1[#2]}] \draw [line
  width=1ex,->] (0,.5ex) -- (2em,.5ex);  \>
  \kern-.5em\tikz [baseline,>={#1[#2]}] \draw [line
  width=1em,->] (0,.5ex) -- (2em,.5ex);  \\
}

\def\arrowexample#1[#2]{\def\temp{#1}\ifx\temp\pgfutil@empty\arrowexample@\currentarrowtype[{#2}]\else\arrowexample@#1[{#2}]\fi}
\def\arrowexample@#1[#2]{%
  {\sfcode`\.1000\small\texttt{#1[#2]}} \>
  \tikz [baseline,>={#1[#2]}] \draw [line
  width=0.4pt,->] (0,.5ex) -- (2em,.5ex); thin \>
  \tikz [baseline,>={#1[#2]}] \draw [line
  width=0.8pt,->] (0,.5ex) -- (2em,.5ex); \textbf{thick} \>
  \tikz [baseline,>={#1[#2]}] \draw [line
  width=1.6pt,->] (0,.5ex) -- (3em,.5ex); \\
}
\def\arrowexampledup[#1]{\arrowexample[{#1] \currentarrowtype[}]}
\def\arrowexampledupdot[#1]{\arrowexample[{#1] . \currentarrowtype[}]}

\def\arrowexampledouble#1[#2]{\def\temp{#1}\ifx\temp\pgfutil@empty\arrowexampledouble@\currentarrowtype[{#2}]\else\arrowexampledouble@#1[{#2}]\fi}
\def\arrowexampledouble@#1[#2]{%
  {\sfcode`\.1000\small\texttt{#1[#2]} on double line} \>
  \tikz [baseline,>={#1[#2]}]
    \draw [double equal sign distance,line width=0.4pt,->] (0,.5ex) -- (2em,.5ex); thin \>
  \tikz [baseline,>={#1[#2]}]
    \draw [double equal sign distance,line width=0.8pt,->] (0,.5ex) -- (2em,.5ex); \textbf{thick} \>
  \tikz [baseline,>={#1[#2]}]
    \draw [double equal sign distance, line width=1.6pt,->] (0,.5ex) -- (3em,.5ex); \\
}



\newcommand\symarrow[1]{%
  \index{#1@\protect\texttt{#1} arrow tip}%
  \index{Arrow tips!#1@\protect\texttt{#1}}%
  \texttt{#1}& yields thick
  \begin{tikzpicture}[arrows={#1-#1},thick,baseline]
    \useasboundingbox (-1mm,-0.5ex) rectangle (1.1cm,2ex);
    \fill [black!15] (1cm,-.5ex) rectangle (1.1cm,1.5ex) (-1mm,-.5ex) rectangle (0mm,1.5ex) ;
    \draw (0pt,.5ex) -- (1cm,.5ex);
  \end{tikzpicture} and thin
  \begin{tikzpicture}[arrows={#1-#1},thin,baseline]
    \useasboundingbox (-1mm,-0.5ex) rectangle (1.1cm,2ex);
    \fill [black!15] (1cm,-.5ex) rectangle (1.1cm,1.5ex) (-1mm,-.5ex) rectangle (0mm,1.5ex) ;
    \draw (0pt,.5ex) -- (1cm,.5ex);
  \end{tikzpicture}
}
\newcommand\symarrowdouble[1]{%
  \index{#1@\protect\texttt{#1} arrow tip}%
  \index{Arrow tips!#1@\protect\texttt{#1}}%
  \texttt{#1}& yields thick
  \begin{tikzpicture}[arrows={#1-#1},thick,baseline]
    \useasboundingbox (-1mm,-0.5ex) rectangle (1.1cm,2ex);
    \fill [black!15] (1cm,-.5ex) rectangle (1.1cm,1.5ex) (-1mm,-.5ex) rectangle (0mm,1.5ex) ;
    \draw (0pt,.5ex) -- (1cm,.5ex);
  \end{tikzpicture}
  and thin
  \begin{tikzpicture}[arrows={#1-#1},thin,baseline]
    \useasboundingbox (-1mm,-0.5ex) rectangle (1.1cm,2ex);
    \fill [black!15] (1cm,-.5ex) rectangle (1.1cm,1.5ex) (-1mm,-.5ex) rectangle (0mm,1.5ex) ;
    \draw (0pt,.5ex) -- (1cm,.5ex);
  \end{tikzpicture}, double
  \begin{tikzpicture}[arrows={#1-#1},thick,baseline]
    \useasboundingbox (-1mm,-0.5ex) rectangle (1.1cm,2ex);
    \fill [black!15] (1cm,-.5ex) rectangle (1.1cm,1.5ex) (-1mm,-.5ex) rectangle (0mm,1.5ex) ;
    \draw[double,double equal sign distance] (0pt,.5ex) -- (1cm,.5ex);
  \end{tikzpicture} and
  \begin{tikzpicture}[arrows={#1-#1},thin,baseline]
    \useasboundingbox (-1mm,-0.5ex) rectangle (1.1cm,2ex);
    \fill [black!15] (1cm,-.5ex) rectangle (1.1cm,1.5ex) (-1mm,-.5ex) rectangle (0mm,1.5ex) ;
    \draw[double,double equal sign distance] (0pt,.5ex) -- (1cm,.5ex);
  \end{tikzpicture}
}

\newcommand\sarrow[2]{%
  \index{#1@\protect\texttt{#1} arrow tip}%
  \index{Arrow tips!#1@\protect\texttt{#1}}%
  \index{#2@\protect\texttt{#2} arrow tip}%
  \index{Arrow tips!#2@\protect\texttt{#2}}%
  \texttt{#1-#2}& yields thick
  \begin{tikzpicture}[arrows={#1-#2},thick,baseline]
    \useasboundingbox (-1mm,-0.5ex) rectangle (1.1cm,2ex);
    \fill [black!15] (1cm,-.5ex) rectangle (1.1cm,1.5ex) (-1mm,-.5ex) rectangle (0mm,1.5ex) ;
    \draw (0pt,.5ex) -- (1cm,.5ex);
  \end{tikzpicture} and thin
  \begin{tikzpicture}[arrows={#1-#2},thin,baseline]
    \useasboundingbox (-1mm,-0.5ex) rectangle (1.1cm,2ex);
    \fill [black!15] (1cm,-.5ex) rectangle (1.1cm,1.5ex) (-1mm,-.5ex) rectangle (0mm,1.5ex) ;
    \draw (0pt,.5ex) -- (1cm,.5ex);
  \end{tikzpicture}
}

\newcommand\sarrowdouble[2]{%
  \index{#1@\protect\texttt{#1} arrow tip}%
  \index{Arrow tips!#1@\protect\texttt{#1}}%
  \index{#2@\protect\texttt{#2} arrow tip}%
  \index{Arrow tips!#2@\protect\texttt{#2}}%
  \texttt{#1-#2}& yields thick
  \begin{tikzpicture}[arrows={#1-#2},thick,baseline]
    \useasboundingbox (-1mm,-0.5ex) rectangle (1.1cm,2ex);
    \fill [black!15] (1cm,-.5ex) rectangle (1.1cm,1.5ex) (-1mm,-.5ex) rectangle (0mm,1.5ex) ;
    \draw (0pt,.5ex) -- (1cm,.5ex);
  \end{tikzpicture} and thin
  \begin{tikzpicture}[arrows={#1-#2},thin,baseline]
    \useasboundingbox (-1mm,-0.5ex) rectangle (1.1cm,2ex);
    \fill [black!15] (1cm,-.5ex) rectangle (1.1cm,1.5ex) (-1mm,-.5ex) rectangle (0mm,1.5ex) ;
    \draw (0pt,.5ex) -- (1cm,.5ex);
  \end{tikzpicture}, double
  \begin{tikzpicture}[arrows={#1-#2},thick,baseline]
    \useasboundingbox (-1mm,-0.5ex) rectangle (1.1cm,2ex);
    \fill [black!15] (1cm,-.5ex) rectangle (1.1cm,1.5ex) (-1mm,-.5ex) rectangle (0mm,1.5ex) ;
    \draw[double,double equal sign distance] (0pt,.5ex) -- (1cm,.5ex);
  \end{tikzpicture} and
  \begin{tikzpicture}[arrows={#1-#2},thin,baseline]
    \useasboundingbox (-1mm,-0.5ex) rectangle (1.1cm,2ex);
    \fill [black!15] (1cm,-.5ex) rectangle (1.1cm,1.5ex) (-1mm,-.5ex) rectangle (0mm,1.5ex) ;
    \draw[double,double equal sign distance] (0pt,.5ex) -- (1cm,.5ex);
  \end{tikzpicture}
}

\newcommand\carrow[1]{%
  \index{#1@\protect\texttt{#1} arrow tip}%
  \index{Arrow tips!#1@\protect\texttt{#1}}%
  \texttt{#1}& yields for line width 1ex
  \begin{tikzpicture}[arrows={#1-#1},line width=1ex,baseline]
    \useasboundingbox (-1mm,-0.5ex) rectangle (1.6cm,2ex);
    \fill [black!15] (1.5cm,-.5ex) rectangle (1.6cm,1.5ex) (-1mm,-.5ex) rectangle (0mm,1.5ex) ;
    \draw (0pt,.5ex) -- (1.5cm,.5ex);
  \end{tikzpicture}
}
\def\myvbar{\char`\|}
\newcommand\plotmarkentry[1]{%
  \index{#1@\protect\texttt{#1} plot mark}%
  \index{Plot marks!#1@\protect\texttt{#1}}
  \texttt{\char`\\pgfuseplotmark\char`\{\declare{\noligs{#1}}\char`\}} &
  \tikz\draw[color=black!25] plot[mark=#1,mark options={fill=examplefill,draw=black}] coordinates{(0,0) (.5,0.2) (1,0) (1.5,0.2)};\\
}
\newcommand\plotmarkentrytikz[1]{%
  \index{#1@\protect\texttt{#1} plot mark}%
  \index{Plot marks!#1@\protect\texttt{#1}}
  \texttt{mark=\declare{\noligs{#1}}} & \tikz\draw[color=black!25]
  plot[mark=#1,mark options={fill=examplefill,draw=black}]
    coordinates {(0,0) (.5,0.2) (1,0) (1.5,0.2)};\\
}



\ifx\scantokens\@undefined
  \PackageError{pgfmanual-macros}{You need to use extended latex
    (elatex) or (pdfelatex) to process this document}{}
\fi

\begingroup
\catcode`|=0
\catcode`[= 1
\catcode`]=2
\catcode`\{=12
\catcode `\}=12
\catcode`\\=12 |gdef|find@example#1\end{codeexample}[|endofcodeexample[#1]]
|endgroup

% define \returntospace.
%
% It should define NEWLINE as {}, spaces and tabs as \space.
\begingroup
\catcode`\^=7
\catcode`\^^M=13
\catcode`\^^I=13
\catcode`\ =13%
\gdef\returntospace{\catcode`\ =13\def {\space}\catcode`\^^I=13\def^^I{\space}}
\gdef\showreturn{\show^^M}
\endgroup

\begingroup
\catcode`\%=13
\catcode`\^^M=13
\gdef\commenthandler{\catcode`\%=13\def%{\@gobble@till@return}}
\gdef\@gobble@till@return#1^^M{}
\gdef\@gobble@till@return@ignore#1^^M{\ignorespaces}
\gdef\typesetcomment{\catcode`\%=13\def%{\@typeset@till@return}}
\gdef\@typeset@till@return#1^^M{{\def%{\char`\%}\textsl{\char`\%#1}}\par}
\endgroup

% Define tab-implementation functions
%   \codeexample@tabinit@replacementchars@
% and
%   \codeexample@tabinit@catcode@
%
% They should ONLY be used in case that tab replacement is active.
%
% This here is merely a preparation step.
%
% Idea:
% \codeexample@tabinit@catcode@ will make TAB active
% and
% \codeexample@tabinit@replacementchars@ will insert as many spaces as
% /codeexample/tabsize contains.
{
\catcode`\^^I=13
% ATTENTION: do NOT use tabs in these definitions!!
\gdef\codeexample@tabinit@replacementchars@{%
 \begingroup
 \count0=\pgfkeysvalueof{/codeexample/tabsize}\relax
 \toks0={}%
 \loop
 \ifnum\count0>0
  \advance\count0 by-1
  \toks0=\expandafter{\the\toks0\ }%
 \repeat
 \xdef\codeexample@tabinit@replacementchars@@{\the\toks0}%
 \endgroup
 \let^^I=\codeexample@tabinit@replacementchars@@
}%
\gdef\codeexample@tabinit@catcode@{\catcode`\^^I=13}%
}%

% Called after any options have been set. It assigns
%   \codeexample@tabinit@catcode
% and
%   \codeexample@tabinit@replacementchars
% which are used inside of
%\begin{codeexample}
% ...
%\end{codeexample}
%
% \codeexample@tabinit@catcode  is either \relax or it makes tab
% active.
%
% \codeexample@tabinit@replacementchars is either \relax or it inserts
% a proper replacement sequence for tabs (as many spaces as
% configured)
\def\codeexample@tabinit{%
  \ifnum\pgfkeysvalueof{/codeexample/tabsize}=0\relax
    \let\codeexample@tabinit@replacementchars=\relax
    \let\codeexample@tabinit@catcode=\relax
  \else
    \let\codeexample@tabinit@catcode=\codeexample@tabinit@catcode@
    \let\codeexample@tabinit@replacementchars=\codeexample@tabinit@replacementchars@
  \fi
}

\newif\ifpgfmanualtikzsyntaxhilighting

\pgfqkeys{/codeexample}{%
  width/.code=  {\setlength\codeexamplewidth{#1}},
  graphic/.code=  {\colorlet{graphicbackground}{#1}},
  code/.code=  {\colorlet{codebackground}{#1}},
  execute code/.is if=code@execute,
  code only/.code=  {\code@executefalse},
  pre/.store in=\code@pre,
  post/.store in=\code@post,
  % #1 is the *complete* environment contents as it shall be
  % typeset. In particular, the catcodes are NOT the normal ones.
  typeset listing/.code=  {#1},
  render instead/.store in=\code@render,
  vbox/.code=  {\def\code@pre{\vbox\bgroup\setlength{\hsize}{\linewidth-6pt}}\def\code@post{\egroup}},
  ignorespaces/.code=  {\let\@gobble@till@return=\@gobble@till@return@ignore},
  leave comments/.code=  {\def\code@catcode@hook{\catcode`\%=12}\let\commenthandler=\relax\let\typesetcomment=\relax},
  tabsize/.initial=0,% FIXME : this here is merely used for indentation. It is just a TAB REPLACEMENT.
  every codeexample/.style={width=4cm+7pt, tikz syntax=true},
  from file/.code={\codeexamplefromfiletrue\def\codeexamplesource{#1}},
  tikz syntax/.is if=pgfmanualtikzsyntaxhilighting,
}


\newread\examplesource


% Opening, reading and closing the results file

\def\opensource#1{
  \immediate\openin\examplesource=#1
}
\def\do@codeexamplefromfile{%
  \immediate\openin\examplesource\expandafter{\codeexamplesource}%
  \def\examplelines{}%
  \readexamplelines
  \closein\examplesource
  \expandafter\endofcodeexample\expandafter{\examplelines}%
}

\def\readexamplelines{
  \ifeof\examplesource%
  \else
    \immediate\read\examplesource to \exampleline
    \expandafter\expandafter\expandafter\def\expandafter\expandafter\expandafter\examplelines\expandafter\expandafter\expandafter{\expandafter\examplelines\exampleline}
    \expandafter\readexamplelines%
  \fi
}

\let\code@pre\pgfutil@empty
\let\code@post\pgfutil@empty
\let\code@render\pgfutil@empty
\def\code@catcode@hook{}

\newif\ifcodeexamplefromfile
\newdimen\codeexamplewidth
\newif\ifcode@execute
\newbox\codeexamplebox
\def\codeexample[#1]{%
  \begingroup%
  \code@executetrue
  \pgfqkeys{/codeexample}{every codeexample,#1}%
  \pgfmanualswitchoncolors%
  \ifcodeexamplefromfile\begingroup\fi
  \codeexample@tabinit% assigns \codeexample@tabinit@[catcode,replacementchars]
  \parindent0pt
  \begingroup%
  \par%
  \medskip%
  \let\do\@makeother%
  \dospecials%
  \obeylines%
  \@vobeyspaces%
  \catcode`\%=13%
  \catcode`\^^M=13%
  \code@catcode@hook%
  \codeexample@tabinit@catcode
  \relax%
  \ifcodeexamplefromfile%
    \expandafter\do@codeexamplefromfile%
  \else%
    \expandafter\find@example%
  \fi}
\def\endofcodeexample#1{%
  \endgroup%
  \ifcode@execute%
    \setbox\codeexamplebox=\hbox{%
      \ifx\code@render\pgfutil@empty%
      {%
        {%
          \returntospace%
          \commenthandler%
          \xdef\code@temp{#1}% removes returns and comments
        }%
        \catcode`\^^M=9%
        \colorbox{graphicbackground}{\color{black}\ignorespaces%
          \code@pre\expandafter\scantokens\expandafter{\code@temp\ignorespaces}\code@post\ignorespaces}%
      }%
      \else%
        \colorbox{graphicbackground}{\color{black}\ignorespaces%
          \code@render}%
      \fi%
    }%
    \ifdim\wd\codeexamplebox>\codeexamplewidth%
      \def\code@start{\par}%
      \def\code@flushstart{}\def\code@flushend{}%
      \def\code@mid{\parskip2pt\par\noindent}%
      \def\code@width{\linewidth-6pt}%
      \def\code@end{}%
    \else%
      \def\code@start{%
        \linewidth=\textwidth%
        \parshape \@ne 0pt \linewidth
        \leavevmode%
        \hbox\bgroup}%
      \def\code@flushstart{\hfill}%
      \def\code@flushend{\hbox{}}%
      \def\code@mid{\hskip6pt}%
      \def\code@width{\linewidth-12pt-\codeexamplewidth}%
      \def\code@end{\egroup}%
    \fi%
    \code@start%
    \noindent%
    \begin{minipage}[t]{\codeexamplewidth}\raggedright
      \hrule width0pt%
      \footnotesize\vskip-1em%
      \code@flushstart\box\codeexamplebox\code@flushend%
      \vskip-1ex
      \leavevmode%
    \end{minipage}%
  \else%
    \def\code@mid{\par}
    \def\code@width{\linewidth-6pt}
    \def\code@end{}
  \fi%
  \code@mid%
  \colorbox{codebackground}{%
    \pgfkeysvalueof{/codeexample/prettyprint/base color}%
    \begin{minipage}[t]{\code@width}%
      {%
        \let\do\@makeother
        \dospecials
        \frenchspacing\@vobeyspaces
        \normalfont\ttfamily\footnotesize
        \typesetcomment%
        \codeexample@tabinit@replacementchars
        \@tempswafalse
        \def\par{%
          \if@tempswa
          \leavevmode \null \@@par\penalty\interlinepenalty
          \else
          \@tempswatrue
          \ifhmode\@@par\penalty\interlinepenalty\fi
          \fi}%
        \obeylines
        \everypar \expandafter{\the\everypar \unpenalty}%
        \pgfkeysvalueof{/codeexample/typeset listing/.@cmd}{#1}\pgfeov
      }
    \end{minipage}}%
  \code@end%
  \par%
  \medskip
  \endcodeexample\endgroup
}

\def\endcodeexample{\endgroup}


\makeatother

\usepackage{pgfmanual}


% autoxref is now always on

% \makeatletter
% % \pgfautoxrefs will be defined by 'make dist'
% \pgfutil@ifundefined{pgfautoxrefs}{%
%   \renewcommand\pgfmanualpdflabel[3][]{#3}% NO-OP
%   \def\pgfmanualpdfref#1#2{#2}%
%   \pgfkeys{
%     /pdflinks/codeexample links=false,% DISABLED.
%   }%
% }{}
% \makeatother

\newdimen\pgfmanualcslinkpreskip

% Styling of the pretty printer
\pgfkeys{
  /codeexample/syntax hilighting/.style={
    /codeexample/prettyprint/key name/.code={\textcolor{keycolor}{\pgfmanualpdfref{##1}{\noligs{##1}}}},
    /codeexample/prettyprint/key name with handler/.code 2 args={\textcolor{keycolor}{\pgfmanualpdfref{##1}{\noligs{##1}}}/\textcolor{blue!70!black}{\pgfmanualpdfref{/handlers/##2}{\noligs{##2}}}},
    /codeexample/prettyprint/key value display only/.code={\textcolor{keycolor}{{\itshape{\let\pgfmanualwordstartup\relax\pgfmanualprettyprintcode{##1}}}}},
    /codeexample/prettyprint/cs/.code={\textcolor{cscolor}{\pgfmanualcslinkpreskip4.25pt\pgfmanualpdfref{##1}{\noligs{##1}}}},
    /codeexample/prettyprint/cs with args/.code 2 args={\textcolor{black}{\pgfmanualcslinkpreskip4.25pt\pgfmanualpdfref{##1}{\noligs{##1}}}\{\textcolor{black}{\pgfmanualprettyprintcode{##2}}\pgfmanualclosebrace},
    /codeexample/prettyprint/cs arguments/pgfkeys/.initial=1,
    /codeexample/prettyprint/cs/pgfkeys/.code 2 args={\textcolor{black}{\pgfmanualcslinkpreskip4.25pt\pgfmanualpdfref{##1}{\noligs{##1}}}\{\textcolor{black}{\pgfmanualprettyprintpgfkeys{##2}}\pgfmanualclosebrace},
    /codeexample/prettyprint/cs arguments/begin/.initial=1,
    /codeexample/prettyprint/cs/begin/.code 2 args={\textcolor{black}{##1}\{\textcolor{cscolor}{\pgfmanualpdfref{##2}{\noligs{##2}}}\pgfmanualclosebrace},
    /codeexample/prettyprint/cs arguments/end/.initial=1,
    /codeexample/prettyprint/cs/end/.code 2 args={\textcolor{black}{##1}\{\textcolor{cscolor}{\pgfmanualpdfref{##2}{\noligs{##2}}}\pgfmanualclosebrace},
    /codeexample/prettyprint/word/.code={\pgfmanualwordstartup{\begingroup\pgfkeyssetvalue{/pdflinks/search key prefixes in}{}\pgfmanualpdfref{##1}{\noligs{##1}}\endgroup}},
    /codeexample/prettyprint/point/.code={\textcolor{pointcolor}{\noligs{##1}}},%
    /codeexample/prettyprint/point with cs/.code 2 args={\textcolor{pointcolor}{(\pgfmanualpdfref{##1}{\noligs{##1}}:\noligs{##2}}},%
    /codeexample/prettyprint/comment font=\itshape,
    /codeexample/prettyprint/base color/.initial=\color{basecolor},
    /pdflinks/render hyperlink/.code={%
      {\setbox0=\hbox{##1}%
        \rlap{{\color{linkcolor}\dimen0\wd0\advance\dimen0by-\pgfmanualcslinkpreskip\hskip\pgfmanualcslinkpreskip\vrule width\dimen0 height-1pt depth1.6pt}}%
        \box0%
      }%
    }
  },/codeexample/syntax hilighting
}

\colorlet{keycolor}{black}
\colorlet{cscolor}{black}
\colorlet{pointcolor}{black}
\colorlet{basecolor}{black}
\colorlet{linkcolor}{black!8}

\def\pgfmanualswitchoncolors{%
  \colorlet{keycolor}{green!50!black}%
  \colorlet{cscolor}{blue!70!black}
  \colorlet{pointcolor}{violet}
  \colorlet{basecolor}{black!55}
  \colorlet{linkcolor}{white}
}

\makeatletter

\def\pgfmanualwordstartup{\textcolor{black}}

\def\noligs#1{\pgfmanualnoligs#1\kern0pt--\pgf@stop}%
\def\pgfmanualnoligs#1--{%
  \pgfutil@ifnextchar\pgf@stop{#1\pgfutil@gobble}{#1-\kern0pt-\kern0pt\pgfmanualnoligs}%
}
\makeatother


%%% Local Variables:
%%% mode: latex
%%% TeX-master: "beameruserguide"
%%% End:

\def\pgfautoxrefs{1}
\def\unitexample#1{%
  \texttt{#1}\indexkey{#1}
  &
  \tikz [baseline,inner sep=0pt] \node[#1=1] {};\\
}


\def\pidlineexample#1{
  \texttt{#1}\indexkey/ISO14617/#1
   &
  \tikz[baseline=-.5ex,circuit pid ISO14617] \draw (0,0) to [#1] (3,0);
}

\newcommand{\pidsymbolexample}[4]{
  \texttt{#1}\index{ISO14617!#1@\protect\texttt{#1}}
  &
  \texttt{#2}\indexkey{#2}
  &
  \texttt{#4}
  &
  \tikz[baseline=-.5ex,circuit pid ISO14617] \draw (0,0) to [#2={#3}] (1.5,0)(3,0); \\
}

\newcommand{\pidmeasurementsymbol}[5]{
  \texttt{#1}\index{Symbols!#1@\protect\texttt{#1}}

  &
  \texttt{#2}\index{ISO14617!#2@\protect\texttt{#2}}
  &
  \texttt{#3}\index{ISO14617!#3@\protect\texttt{#3}}
  &
  \texttt{#4}\index{ISO14617!#4@\protect\texttt{#4}}
  &
  \texttt{#5}\index{ISO14617!#5@\protect\texttt{#5}} \\ \hline
}
\def\empty{}

\makeatletter
\def\index@prologue{\section*{Index}\addcontentsline{toc}{section}{Index}}

\renewcommand*\l@section[2]{%
  \ifnum \c@tocdepth >\z@
    \addpenalty\@secpenalty
    \addvspace{1.0em \@plus\p@}%
    \setlength\@tempdima{2.5em}%
    \begingroup
      \parindent \z@ \rightskip \@pnumwidth
      \parfillskip -\@pnumwidth
      \leavevmode \bfseries
      \advance\leftskip\@tempdima
      \hskip -\leftskip
      #1\nobreak\hfil \nobreak\hb@xt@\@pnumwidth{\hss #2}\par
    \endgroup
  \fi}
\renewcommand*\l@subsection{\@dottedtocline{2}{2.5em}{3.3em}}
\renewcommand*\l@subsubsection{\@dottedtocline{3}{5.8em}{4.2em}}
\def\@pnumwidth{2.2em}
\makeatother

\RequirePackage{pgfmanual}
\pgfkeys{
    % whenever an unqualified key is found, the following key prefix
    % list is tried to find a match.
    /pdflinks/search key prefixes in={/tikz/,/pgf/},
    %
    % the link prefix written to the pdf file:
    /pdflinks/internal link prefix=pgfp,
    %
    /pdflinks/codeexample links=true,
    /pdflinks/warnings=false,   % for debugging
    /pdflinks/show labels=false,% for debugging
}

% Global styles:
\tikzset{
  every plot/.style={prefix=plots/pgf-},
  shape example/.style={
    color=black!30,
    draw,
    fill=yellow!30,
    line width=.5cm,
    inner xsep=2.5cm,
    inner ysep=0.5cm}
}

\makeindex

%%%%%%%%%%%%%%%%%%%%%%%%%%%%%%%%%%%%%%%%%%%%%%%%%%%%%%%%%%%%%%%%%%%%%
%
% This file really starts here, all the above is for pgfmanual
% compatibility
% \usetikzlibrary{tikzlibrarycircuits.pid}
\tikzset{circuit symbol unit=2.5mm}
\sloppy
\def\d{2}
% counter for automatic placement
\newcount\xpos
\newcount\ypos

\begin{document}
\title{Typesetting a Process and Instrumentation Diagram with \LaTeX{} and \tikzname{}}
\author{Jelle Spijker\\e-mail: \texttt{spijker.jelle@gmail.com}}
\date{Version: 1.0, Version date: 02/11/2018}

\maketitle

\section{Abstract}
This package contains a collection of symbols, intended to be used with a \emph{Process \& Instrumentation Diagram}. The symbols are meant to be in agreement with international standard ISO-14617. It extends and modifies, when needed, the \tikzname-library |circuits|. A drafting class is also included, which allows a \tikzname-library |circuits.pid| diagrams, to be presented as a standalone document.


\section{Introduction}
P\&IDs are used in a variety a industries, to design and illustrate the workings of various processes. The purpose of ISO 14617 in its final form is the creation of a library of harmonized graphical symbols for diagrams used in technical applications. Their work has been, and will be, performed in close cooperation between ISO and IEC. The ultimate result is intended to be published as a standard common to ISO and IEC, which their technical committees responsible for specific application fields can use in preparing International Standards and manuals. This \tikzname-library is a translation of these norms, such that the can be used in \LaTeX documents.

The author has access to the current ISO version, via his employer \textbf{Royal IHC}, these norms were bought by them and cannot be redistribute to the general public. However all individual ISO-14617 symbols can also be found at \href{https://www.iso.org/obp/ui#search}{ISO.org}, just search for 14617 and go to the graphical symbols tab.

\bigskip
\noindent The following sub norms are used for this \tikzname-library:
\begin{itemize}
 \setlength\itemsep{0pt}
 \item 14617-1 General information and indexes
 \item 14617-2 Symbols having general application
 \item 14617-3 Connections and related devices
 \item 14617-4 Actuators and related devices
 \item 14617-5 Measurement and control devices
 \item 14617-6 Measurement and control functions
 \item 14617-7 Basic mechanical components
 \item 14617-8 Valves and dampers
 \item 14617-9 Pumps, compressors and fans
 \item 14617-10 Fluid power converters
 \item 14617-11 Devices for heat transfer and heat engines
 \item 14617-12 Devices for separating, purification and mixing
 \item 14617-13 Devices for material processing
 \item 14617-14 Devices for transport and handling of material
 \item 14617-15 Installation diagrams and network maps
\end{itemize}

\section{Example}
\begin{codeexample}[]
\begin{tikzpicture}[circuit pid ISO14617, every info/.style={black, font=\tiny}]
\begin{complex device}[name=solar absorption cooler, xshift=-160mm, yshift=100mm, show border, show label]
\coordinate (solar absorption cooler base) at (0,0);
\node[envelope={name=SC1, light, quantity transition to higher, info={[xshift=-2, yshift=-10]T},
info'={[xshift=10, yshift=30]\( SC_{1} \)}}] at (solar absorption cooler base) {};
\node[pressure vessel={name=T1, info={[yshift=-15]right:\( T_{1} \)}, rotate=90, anchor=north,
   circuit symbol size=width 5 height 3}] at ($(SC1.output)+(\d,0)$) {};
\node[pump={name=P1, displacement, info=\( P_{1} \)}] at ($(SC1)!0.5!(T1)-(0,\d)$) {};
\node[envelope={name=G1, with={heating coil}{0}{0},info={[xshift=10]\( G_{1} \)}}] at ($(T1)+(2*\d,0)$) {};
\node[pump={name=P2, displacement, info=\( P_{2} \)}] at ($(T1)!0.5!(G1)+(0,\d)$) {};
\node[envelope={name=R1, device for separating, info={[yshift=-15]CH}, info'=above:\( R_{1} \)}]
at ($(G1)+(\d,0)$) {};
\node[envelope={name=C1, with={heating coil}{0}{0},info={[xshift=-5]\( C_{1} \)}}] at ($(R1)+(\d,0)$) {};
\node[valve={name=V1, info={\( V_{1} \)}, rotate=90}] at ($(C1.south)-(-\d,2*\d)$) {};
\node[envelope={name=E1, with={heating coil}{0}{0},info={[xshift=-5]\( E_{1} \)}}] at ($(V1)-(\d,2*\d)$) {};
\node[cooling tower={name=CT1, with={spray nozzle}{0}{2.5}, info=\( CT_{1} \)}] at ($(C1)+(2*\d,0)$) {};
\node[pump={name=P3, displacement, rotate=-90, info={\( P_{3} \)}}] at (CT1 |- V1) {};
\node[coil shaped heat exchanger={name=H1, rotate=90, info=\( H_{1} \), huge circuit symbols},
 anchor=refrigerant in] at ($(G1)-(0,\d+0.125*\d)$) {};
\node[pump={name=P4, displacement, info=\( P_{4} \), rotate=90}] at (V1 -| H1.input) {};
\node[valve={name=V2, info=\( V_{2} \), rotate=90}] at (P4 -| H1.refrigerant in) {};
\node[envelope={name=A1, with={heating coil}{0}{0}, info={[xshift=5]\( A_{1} \)}}] at ($(V2)-(0,\d)$) {};

% === Solar refrigerant system ===
\begin{flow system}[color=red!80!white, id=010]
\draw[flow path, id=P001, show'] (T1.north) to (SC1);
\draw[flow path, id=P002, show, pos=0.8] (SC1) to (P1 -| SC1) to (P1);
\draw[flow path, id=P003, show, pos=0.2] (P1) to (P1 -| T1) to (T1.west);
\draw[flow path, id=P004, show', pos=0.85] (G1-heating coil.south) to ($(G1-heating coil.south -| T1.south)$);
\draw[flow path, id=P005, show, pos=0.75] (T1) to (T1 |- P2) to (P2);
\draw[flow path, id=P006, show, pos=0.3] (P2) to (P2 -| G1) to (G1-heating coil.north);
\end{flow system}

% == Cooling water system ===
\begin{flow system}[id=020, color=blue!80]
\draw[flow path, id=P001, show] (C1-heating coil.north) to (CT1-spray nozzle.north -| C1)
to (CT1-spray nozzle.north);
\draw[flow path, id=P002, show'] (CT1.south) to (P3);
\draw[flow path, id=P003, show'] (P3) to (P3 |- A1-heating coil.south) to (A1-heating coil.south);
\draw[flow path, id=P004, show, pos=0.3] (A1-heating coil.north) to (A1-heating coil.north -| C1)
to (C1-heating coil.south);
\end{flow system}

% === NH3 refrigerant
\begin{flow system}[id=030, color=red!50!black]
\draw[flow path, id=P001, show] (R1) to (C1);
\draw[flow path, id=P002, show', pos=0.75] (C1) to (C1 -| V1) to (V1);
\draw[flow path, id=P004, show', pos=0.85] (V1) to (V1 |- E1) to (E1);
\draw[flow path, id=P005, show'] (E1) to (E1 -| A1) to (A1);
\crossings{020P003}{030P004}
\end{flow system}

% === NH3-H20 refrigerant
\begin{flow system}[id=040, color=green!50!blue]
\draw[flow path, id=P001, show] (G1) to (R1);
\draw[flow path, id=P002, show'] (R1) to ++(0,-0.75\d) to ([xshift=5]\currentcoordinate -| G1)
to ([xshift=5]G1.south);
\draw[flow path, id=P003, show'] (H1.refrigerant in) to (V2);
\draw[flow path, id=P004, show'] (V2) to (A1);
\draw[flow path, id=P005, show, pos=0.75] (A1) to (A1 -| P4) to (P4);
\draw[flow path, id=P006, show] (P4) to (H1.input);
\draw[flow path, id=P007, show] (H1.output) to (G1.south -| H1.output);
\draw[flow path, id=P008, show', pos=0.8] (G1) to (G1 -| H1.refrigerant out) to (H1.refrigerant out);
\crossings{010P004}{040P008};
\end{flow system}
\end{complex device}
\end{tikzpicture}
\end{codeexample}

\newpage
\section{P\&ID library}
The |circuit pid| libraries can be used to draw different kind of \emph{Process \& Instrumentation Diagrams}. It is not a single library, but a hierarchy of libraries that work in concert. The main goal was to convert the ISO-14617 symbols and drawing conventions to a \LaTeX / \tikzname\ environment. This library is setup in such a way that it can easily be extended with different drawing conventions and norms.

A P\&ID circuit typically consist of numerous elements, such as pumps, valves, heaters, actuators, sensors, tanks etc; which are connected through a flow of a certain quantity, usually: mass, volume, electrical charge or information. In
\pgfname/\tikzname, we use nodes for these elements and normal lines for the  flow of a quantity. \tikzname\ offers a large number of different ways of positioning and connecting nodes in general, all of which can be used here. Additionally, the |circuits.pid| library defines additional useful commands, such as |to|-path, |with|-node and |at|-node. Those are useful for elements such as actuators en sensors, or for the combination of various nodes configurations.

A simple example shows how a P\&ID could be created. A user has to decide which symbol graphics he would like to use. Which only option (at the time) is the ISO-14617 norm at the moment. By include the library |circuits.pid.ISO14617| in the preamble of his document.
\begin{codeexample}[code only]
 \usetikzlibrary{circuits.pid.ISO14617}
\end{codeexample}
To create a \tikzname-picture illustrating the process of pumping a fluid between two tanks the following code suffices.
\begin{codeexample}[]
 \begin{tikzpicture}[circuit pid ISO14617]
  \node[tank={name=T1}] at (0,0) {};
  \node[tank={name=T2}] at (3,0) {};
  \draw (T1) to [pump] (T2);
 \end{tikzpicture}
\end{codeexample}
The example above can be extended with the use of \emph{annotations}, these are extensions for nodes, that give supplementary information. The pump used in the previous example can easily desribed as a displacement pump, with the use of annotations.
\begin{codeexample}[]
 \begin{tikzpicture}[circuit pid ISO14617]
  \draw (0,0) to [pump=displacement] (2,0);
 \end{tikzpicture}
\end{codeexample}

\subsection{The Base Circuit Library}
The following sections is directly copied out of the \pgfname/\tikzname-manual, such that this manual can be used as a stand-alone document. The code and examples are converted from electrical type circuits to  P\&ID circuits.

\begin{tikzlibrary}{circuits}
 This library is a base library that is included by other circuit
 libraries. You do not include it directly, but you will typically
 use some of the general keys, described below.
\end{tikzlibrary}

\begin{key}{/tikz/circuits}
 This key should be passed as an option to a picture or a scope that
 contains a circuit. It will do some internal setups. This key is
 normally called by more specialized keys like |circuit ee IEC| or |circuit pid ISO14617|.
\end{key}

\subsubsection{Symbol Size}

\begin{key}{/tikz/circuit symbol unit=\meta{dimension} (initially 7pt)}
 This dimension is a ``unit'' for the size of symbols. The libraries
 generally define the sizes of symbols relative to this
 dimension. For instance, the symbol of a pump, by
 default, in the ISO14617 library equal to two and a half times this
 \meta{dimension}. When you change this \meta{dimension}, the size of
 all symbols will automatically change accordingly.

 Note, that it is still possible to overwrite the size of any
 particular symbol. These settings apply only to the default sizes.

 \begin{codeexample}[]
  \begin{tikzpicture}[circuit pid ISO14617]
   \draw (0,1) to [valve] (3.5,1);
   \draw[circuit symbol unit=14pt]
   (0,0) to [valve] (3.5,0);
  \end{tikzpicture}
 \end{codeexample}
\end{key}

\begin{stylekey}{/tikz/huge circuit symbols}
 This style sets the default circuit symbol unit to |10pt|.
\end{stylekey}
\begin{stylekey}{/tikz/large circuit symbols}
 This style sets the default circuit symbol unit to |8pt|.
\end{stylekey}
\begin{stylekey}{/tikz/medium circuit symbols}
 This style sets the default circuit symbol unit to |7pt|.
\end{stylekey}
\begin{stylekey}{/tikz/small circuit symbols}
 This style sets the default circuit symbol unit to |6pt|.
\end{stylekey}
\begin{stylekey}{/tikz/tiny circuit symbols}
 This style sets the default circuit symbol unit to |5pt|.
\end{stylekey}

\begin{key}{/tikz/circuit symbol size=|width| \meta{width} |height|
  \meta{height}}
 This key sets |minimum height| to \meta{height} times the current
 value of the circuit symbol unit and the |minimum width| to
 \meta{width} times this value. Thus, this option can be used with a
 node command to set the size of the node as a multiple of the
 circuit symbol unit.

 \begin{codeexample}[]
  \begin{tikzpicture}[circuit pid ISO14617]
   \draw (0,1) to [valve] (2,1) to[pump] (4,1);

   \begin{scope}
    [every valve/.style={circuit symbol size=width 4 height 1}]
    \draw (0,0) to [valve] (2,0) to[pump] (4,0);
   \end{scope}
  \end{tikzpicture}
 \end{codeexample}
\end{key}

\subsubsection{Declaring New Symbols}

\begin{key}{/tikz/circuit declare symbol=\meta{name}}
 This key is used to declare a symbol. It does not cause this symbol
 to be shown nor does it set a graphic to be used for the symbol, it
 simply ``prepares'' several keys that can later be used to draw a
 symbol and to configure it.

 In detail, the first key that is defined is just called
 \meta{name}. This key should be given as an option to a |node| or on
 a |to| path, as explained below. The key will take options, which
 can be used to influence the way the symbol graphic is rendered.

 Let us have a look at an example. Suppose we want to define a symbol
 called |foo|, which just looks like a simple rectangle. We could
 then say
 \begin{codeexample}[code only]
  \tikzset{circuit declare symbol=foo}
 \end{codeexample}
 The symbol could now be used like this:
 \begin{codeexample}[code only]
  \node [foo]       at (1,1) {};
  \node [foo={red}] at (2,1) {};
 \end{codeexample}

 However, in the above example we would not actually see anything
 since we have not yet set up the graphic to be used by |foo|. For
 this, we must use a key called |set foo graphic| or, generally,
 |set| \meta{name} |graphic|. This key gets graphic options as parameter
 that will be set when a symbol |foo| should be shown:
 \begin{codeexample}[]
  \begin{tikzpicture}
   [circuit pid ISO14617,
    circuit declare symbol=foo,
    set foo graphic={draw,shape=rectangle,minimum size=5mm}]

   \node [foo]       at (1,1) {};
   \node [foo={red}] at (2,1) {};
  \end{tikzpicture}
 \end{codeexample}

 In detail, when you use the key \meta{name}=\meta{options} with a
 node, the following happens:
 \begin{enumerate}
  \item The |inner sep| is set to |0.5pt|.
  \item The following style is executed:
        \begin{stylekey}{/tikz/every circuit symbol}
         Use this style to set up things in general.
        \end{stylekey}
  \item The graphic options that have been set using
        |set| \meta{name} |graphic| are set.
  \item The style |every |\meta{name} is executed. You can use it to
        configure the symbol further.
  \item The \meta{options} are executed.
 \end{enumerate}

 The key \meta{name} will have a different effect when it is used on
 a |to| path command inside a |circuit| environment (the |circuit|
 environment sets up |to| paths in such a way that the use of a key
 declared using |circuit declare symbol| is automatically detected).
 When \meta{name} is used on a |to| path, the above actions also
 happen (setting the inner separation, using the symbol graphic, and
 so on), but they are passed to the key |circuit handle symbol|,
 which is explained next.
\end{key}

\begin{key}{/tikz/circuit handle symbol=\meta{options}}
 This key is mostly used internally. Its purpose is to render a
 symbol. The effect of this key differs, depending on whether it is
 used as the optional argument of a |to| path command or elsewhere.

 If the key is not used as an argument of a |to| path command, the
 \meta{options} are simply executed.

 The more interesting case happens when the key is given on a |to|
 path command. In this case, several things happen:
 \begin{enumerate}
  \item The |to| path is locally changed and set to an internal
        path (which you should not try to change) that consists mostly of
        a single straight line.
  \item The \meta{options} are tentatively executed with filtering
        switched on. Everything is filtered out, except for the key |pos|
        and also the styles |at start|, |very near start|, |near start|,
        |midway|, |near end|, |very near end|, and |at end|. If none of
        them is found, |midway| is used.
  \item The filtered option is used to determine a position for the
        symbol on the path. At the given position (with |pos=0|
        representing the start and |pos=1| representing the end), a node
        will be added to the path (in a manner to be described presently).
  \item This node gets \meta{options} as its option list.
  \item The node is added by virtue of a special |markings|
        decoration. This means that a |mark| command is executed that
        causes the node to be placed as a mark on the path.
  \item The marking decoration will automatically subdivide the path
        and cause a line to be drawn from the start of the path to the
        node's border (at the position that lies on a line from the node's
        center to the start of the path) and then from the node's border
        (at a position on the other side of the node) to the end of the
        path.
  \item The marking decoration will also take care of the case that
        multiple marks are present on a path, in this case the lines from
        and to the borders of the nodes are only between consecutive
        nodes.
  \item The marking decoration will also rotate the coordinate system
        in such a way that the $x$-axis points along the path. Thus, if
        you use the |transform shape| option, the node will ``point
        along'' the path.
  \item In case a node is at |pos=0| or at |pos=1| some special code
        will suppress the superfluous lines to the start or end of the
        path.
 \end{enumerate}
\end{key}

\subsubsection{Pointing Symbols in the Right Direction}

Unlike normal nodes, which generally should not be rotated since this
will make their text hard to read, symbols often need to be
rotated. There are two ways of achieving such rotations:

\begin{enumerate}
 \item When you place a symbol on a |to| path, the graphic symbol is
       automatically rotated such that it ``points along the path.'' Here
       is an examples that shows how the pump shape is automatically rotated around:
       \begin{codeexample}[]
        \tikz [circuit pid ISO14617]
        \draw (3,0) to[valve] (1,0) to[pump] (0,2);
       \end{codeexample}
 \item Many shapes cannot be placed ``on'' a path in this way, namely
       whenever there are more than two possible inputs. Also, you may wish
       to place the nodes first, possibly using a matrix, and connect them
       afterwards. In this case, you can simply add rotations like
       |rotate=90| to the shapes to rotate them. The following four keys
       make this slightly more convenient:
       \begin{key}{/tikz/point up}
        This is the same as |rotate=90|.
        \begin{codeexample}[]
         \tikz [circuit pid ISO14617] \node [pump,point up] {};
        \end{codeexample}
       \end{key}
       \begin{key}{/tikz/point down}
        This is the same as |rotate=-90|.
        \begin{codeexample}[]
         \tikz [circuit pid ISO14617] \node [pump,point down] {};
        \end{codeexample}
       \end{key}
       \begin{key}{/tikz/point left}
        This is the same as |rotate=-180|.
        \begin{codeexample}[]
         \tikz [circuit pid ISO14617] \node [pump,point left] {};
        \end{codeexample}
       \end{key}
       \begin{key}{/tikz/point right}
        This key has no effect.
        \begin{codeexample}[]
         \tikz [circuit pid ISO14617] \node [pump,point right] {};
        \end{codeexample}
       \end{key}
\end{enumerate}


\subsubsection{Info Labels}

Info labels are used to add text to a circuit symbol. Unlike normal
nodes like a rectangle, circuit symbols typically do not have text
``on'' them, but the text is placed next to them (like the text
``$\SI{3}{\meter\per\second}$'' next to a pump).

\tikzname\ already provides the |label| option for this purpose. The
|info| option is built on top of this option, but it comes in some
predefined variants that are especially useful in conjunction with
circuits.

\begin{key}{/tikz/info=\opt{|[|\meta{options}|]|\meta{angle}|:|}\meta{text}}
 This key has nearly the same effect as the |label| key, only the
 following style is used additionally automatically:
 \begin{stylekey}{/tikz/every info}
  Set this style to configure the styling of info labels. Since this
  key is \emph{not} used with normal labels, it provides an easy way
  of changing the way info labels look without changing other
  labels.
 \end{stylekey}
 The \meta{options} and \meta{angle} are passed directly to the
 |label| command.
 \begin{codeexample}[]
  \begin{tikzpicture}[circuit pid ISO14617, every info/.style=red]
   \node [valve,info=$\SI{3}{\meter\per\second}$] {};
  \end{tikzpicture}
 \end{codeexample}

 Hint: To place some text \emph{on} the main node, use |center| as
 the \meta{angle}:
 \begin{codeexample}[]
  \begin{tikzpicture}[circuit pid ISO14617,every info/.style=red]
   \node [pump,info=center:$\SI{3}{\meter\per\second}$] {};
   \node [pump,point up,info=center:$P_1$] at (2,0) {};
  \end{tikzpicture}
 \end{codeexample}
\end{key}

\begin{key}{/tikz/info'=\opt{|[|\meta{options}|]|\meta{angle}|:|}\meta{text}}
 This key works exactly like the |info| key, only in case the
 \meta{angle} is missing, it defaults to |below| instead of the
 current value of  |label position|, which is usually |above|. This
 means that when you use |info|, you get a label above the node,
 while when you use the |info'| key you get a label below the
 node. In case the node has been rotated, the positions of the info
 nodes are rotated accordingly.
 \begin{codeexample}[]
  \begin{tikzpicture}[circuit pid ISO14617,every info/.style=red]
   \draw (0,0) to[valve={info={$\SI{3}{\meter\per\second}$},info'={$V_1$}}] (3,0)
   to[pump={info={$\SI{3}{\meter\per\second}$},info'={$P_1$}}] (3,2);
  \end{tikzpicture}
 \end{codeexample}
\end{key}

\begin{key}{/tikz/info sloped=\opt{|[|\meta{options}|]|\meta{angle}|:|}\meta{text}}
 This key works like |info|, only the |transform shape| option is set
 when the label is drawn, causing it to follow the sloping of the
 main node.
 \begin{codeexample}[]
  \begin{tikzpicture}[circuit pid ISO14617,every info/.style=red]
   \draw (0,0) to[valve={info sloped={$\SI{3}{\meter\per\second}$}}] (3,0)
   to[valve={info sloped={$\SI{3}{\meter\per\second}$}}] (3,2);
  \end{tikzpicture}
 \end{codeexample}
\end{key}

\begin{key}{/tikz/info' sloped=}
 This is a combination of |info'| and |info sloped|.
 \begin{codeexample}[]
  \begin{tikzpicture}[circuit pid ISO14617,every info/.style=red]
   \draw (0,0) to[valve={info' sloped={$\SI{3}{\meter\per\second}$}}] (3,0)
   to[valve={info' sloped={$\SI{3}{\meter\per\second}$}}] (3,2);
  \end{tikzpicture}
 \end{codeexample}
\end{key}

\begin{key}{/tikz/circuit declare unit=\marg{name}\marg{unit}}
 This key is used to declare keys that make it easy to attach
 physical units to nodes. The idea is that instead of
 |info=$\SI{3}{\meter\per\second}$| you can write |speed=3| or instead of
 |info'=$\SI{1}{\cubic\meter\per\second}$| you can write |volume flow'=5|. Although units can be described as |$\frac{m^3}{s}$| it is good practice to make us of the |siunitx| package.

 In detail, four keys are defined, namely |/tikz/|\meta{name},
 |/tikz/|\meta{name}|'|, |/tikz/|\meta{name} |sloped|, and
 |/tikz/|\meta{name}|'| |sloped|. The arguments of all of these keys
 are of the form
 \opt{|[|\meta{options}|]|\meta{angle}|:|}\meta{value} and it is
 passed (slightly modified) to the corresponding key |info|, |info'|,
 |info| |sloped|, or |info'| |sloped|. The ``slight modification'' is the
 following: The text that is passed to the, say, |info| key is not
 \meta{value}, but rather |$\mathrm{|\meta{value}\meta{unit}|}$|
 This means that after you said |circuit declare unit={speed}{\si{\meter\per\second}}|,
 then |speed=5| will have the same effect as
 |info={[every ohm]$\mathrm{5\si{\meter\per\second}}$}|. Here, |every speed| is a  style that allows you to configure the appearance of this unit.  Since the |info| key is used internally, by changing the  |every info| style, you can change the appearance of all units infos.
 \begin{codeexample}[]
  \begin{tikzpicture}[circuit pid ISO14617,
    circuit declare unit={my speed}{\si{\kilo\meter\per\hour}}]
   \draw (0,0) to[valve={my speed' sloped=3}] (3,2);
  \end{tikzpicture}
 \end{codeexample}
\end{key}
\medskip
\noindent The |circuit.pid| library predefines the following unit keys:
\medskip

\begin{tabu}{p{5cm} c}
 \emph{Key} & \emph{Appearance of $1$ unit} \\[.25em]
 \unitexample{/tikz/speed}
 \unitexample{/tikz/mass flow}
 \unitexample{/tikz/volume flow}
\end{tabu}

\subsection{The P\&ID library}
\begin{tikzlibrary}{circuits.pid}
 This library defines general keys for the creation of \emph{Process \& Instrumentation Diagrams}. This library is written as an extension of \tikzname-library |circuits|. It allows for the creation of new type of circuit diagram, using the same syntax and conventions established by \tikzname-library |circuits|. The reader is urged to read the Section~``Circuit Libraries'' of \tikzname\ manual.
\end{tikzlibrary}

\noindent The following keys are defined:
\begin{key}{/tikz/circuit pid}
 This key should be passed as an option to a picture or a scope that contains a diagram. It will do some internal setups. At the time of writing only ISO-14617 P\&ID are implemented. You normally do not use this library directly since it does not
 define any symbol graphics; This is done with the sublibrary |circuits.pid.ISO14617|
\end{key}

\begin{key}{/tikz/measure=\opt{\meta{options}}}
 A

 \tabulinesep=1mm
 \begin{tabu}{|l c | X[l] | X[l] | X[l] |}\hline
  \rowfont{\bfseries}Symbol &  & Measured or initiating variable & Modifier            & Function                               \\ \hline
  \pidmeasurementsymbol{1051}{A}{}{}{alarming}
  \pidmeasurementsymbol{1052}{B}{}{}{displaying discrete state}
  \pidmeasurementsymbol{1053}{C}{}{}{controlling}
  \pidmeasurementsymbol{1054}{D}{density}{difference}{}
  \pidmeasurementsymbol{1055}{E}{electric variable}{}{sensing}
  \pidmeasurementsymbol{1056}{F}{flow rate}{ratio, fraction}{}
  \pidmeasurementsymbol{1057}{G}{gauge, position, length}{}{viewing}
  \pidmeasurementsymbol{1058}{H}{hand}{}{}
  \pidmeasurementsymbol{1059}{I}{}{}{indicating}
  \pidmeasurementsymbol{1060}{J}{power}{scanning}{}
  \pidmeasurementsymbol{1061}{K}{time}{time rate of change}{}
  \pidmeasurementsymbol{1062}{L}{level}{}{}
  \pidmeasurementsymbol{1063}{M}{moisture, humidity}{momentarily}{}
  \pidmeasurementsymbol{1064}{N}{user's choice}{}{user's choice}
  \pidmeasurementsymbol{1065}{O}{user's choice}{}{}
  \pidmeasurementsymbol{1066}{P}{pressure, vacuum}{}{connection of test point}
  \pidmeasurementsymbol{1067}{Q}{quality}{integral, total}{integrating, summing}
  \pidmeasurementsymbol{1068}{R}{radiation}{}{registering, recording}
  \pidmeasurementsymbol{1069}{S}{speed, frequency}{}{switching}
  \pidmeasurementsymbol{1070}{T}{temperature}{}{transmitting}
  \pidmeasurementsymbol{1071}{U}{multi-variable}{}{multi-function}
  \pidmeasurementsymbol{1072}{V}{user's choice}{}{impact on process by valve, pump, etc.}
  \pidmeasurementsymbol{1073}{W}{weight, force}{Multiplying}{}
  \pidmeasurementsymbol{1074}{X}{unclassified}{}{unclassified}
  \pidmeasurementsymbol{1075}{Y}{user's choice}{}{converting, computing }
  \pidmeasurementsymbol{1076}{Z}{number of events, quantity}{}{emergency or safety acting}
 \end{tabu}


\end{key}

\begin{key}{/tikz/at}
 test
\end{key}

\begin{key}{/tikz/with}
 test
\end{key}

\begin{decoration}{/tikz/flow path}
 test
 \begin{key}{/tikz/id}
  test
 \end{key}

 \begin{key}{/tikz/show}
  test
 \end{key}

 \begin{key}{/tikz/show'}
  test
 \end{key}

 \begin{key}{/tikz/pos}
  test
 \end{key}

 \begin{key}{/tikz/pos slope}
  test
 \end{key}

\end{decoration}

\subsection{Implementation: The P\&ID-symbols shape library}
\begin{pgflibrary}{shapes.gates.pid}
 This library defines basic shapes that can be used by all pid-circuit
 libraries. Currently, it defines the following shapes:
 \begin{itemize}
  \item |rectangle pid|
  \item |circle pid|
  \item |direction pid|
 \end{itemize}
 Additionally, the library defines the following arrow tip:
 The |direction pid| arrow tip is basically the same as a |triangle 45|
 arrow tip with rounded joins.

 \begin{tabu}{l l}
  \symarrow{direction pid}
 \end{tabu}

 However, unlike normal arrow tips, its size does \emph{not} depend on
 the current line width. Rather, it depends on the value of its
 arrow options, which should be set to the desired size. Thus, you
 should say something like |\pgfsetarrowoptions{direction pid}{5pt}| to
 set the size of the arrow.
\end{pgflibrary}

\begin{shape}{rectangle pid}
 This shape is completely identical to a normal |rectangle|, only
 there are two additional anchors: The |input| anchor is an alias for
 the |west| anchor, while the |output| anchor is an alias for the
 |east| anchor.
\end{shape}

\begin{shape}{circle pid}
 Like the |rectangle pid| shape, only for circles.
\end{shape}

\begin{shape}{direction pid}
 This shape is rather special. It is intended to be used to ``turn an
 arrow tip into a shape.'' First, you should set the following key to
 the name of an arrow tip:
 \begin{key}{/pgf/direction pid arrow=\meta{right arrow tip name}}
  The value of this key will be used for the arrow tip depicted in
  an |direction pid| shape.
 \end{key}
 When a node of shape |direction pid| is created, several things
 happen:
 \begin{enumerate}
  \item The size of the shape is computed according to the following
        rules: The width of the shape is set up so that the left border of
        the shape is at the left end of the arrow tip and the right border
        is at the right end of the arrow tip. These left and right
        ``ends'' of the arrow are the tip end and the back end specified
        by the arrow itself. You usually need not worry about this width
        setting.

        By comparison, the height of the arrow is given by the current
        setting of |minimum height|. Thus, this key must have been set up
        correctly to reflect the ``real'' height of the arrow tip. The
        reason is that the height of an arrow is not specified when arrows
        are declared and is, thus, not available, here.

        Possibly, the height computation will change in the future to
        reflect the real height of the arrow, so you should generally
        set up the |minimum height| to be the same as the real height.
  \item A straight line from left to right inside the shape's
        boundaries is added to the background path.
  \item The arrow tip, pointing right, is drawn before the background
        path.
 \end{enumerate}
 The anchors of this shape are just the compass anchors, which lie on
 a rectangle whose width and height are the above-computed height and
 width.

 \begin{codeexample}[]
  \begin{tikzpicture}
   \pgfsetarrowoptions{direction pid}{6cm}
   \node[name=s,shape=direction pid,shape example,minimum height=0.7654*6cm] {};
   \foreach \anchor/\placement in
    {center/above, 30/above right,
     north/above, south/below, east/left, west/right,
     north east/above, south east/below, south west/below, north west/above,
     input/left,output/right}
   \draw[shift=(s.\anchor)] plot[mark=x] coordinates{(0,0)}
   node[\placement] {\scriptsize\texttt{(s.\anchor)}};
  \end{tikzpicture}
 \end{codeexample}

 \begin{codeexample}[]
  \begin{tikzpicture}[direction pid arrow=angle 45]
   \node[name=s,shape=direction pid,shape example,minimum height=1.75cm] {};
   \foreach \anchor/\placement in {north/above, south/below,
     output/right, input/left}
   \draw[shift=(s.\anchor)] plot[mark=x] coordinates{(0,0)}
   node[\placement] {\scriptsize\texttt{(s.\anchor)}};
  \end{tikzpicture}
 \end{codeexample}
\end{shape}

\subsection{Implementation: The ISO-14617 style P\&ID-symbol shape library}

\begin{pgflibrary}{shapes.gates.pid.ISO14617}
 This library defines shapes for depicting pid symbols according to
 the ISO14617 recommendations. These shapes will typically be
 used in conjunction with the graphic mechanism detailed earlier, but
 you can also used them directly.
\end{pgflibrary}

\begin{shape}{generic circle ISO14617}
 This shape inherits from |circle pid|, which in turn is just a normal
 |circle| with additional |input| and |output| anchors at the left
 and right ends. However, additionally, this shape allows you to
 specify a path that should be added before the background path using
 the following key:
 \begin{key}{/pgf/generic circle ISO14617/before background=\meta{code}}
  When a node of shape |generic circle ISO14617| is created, the current
  setting of this key is used as the ``before background path.''
  This means that after the circle's background has been
  drawn/filled/whatever, the \meta{code} is executed.

  When the \meta{code} is executed, the coordinate system will have
  been transformed in such a way that the point
  $(1\mathrm{pt},0\mathrm{pt})$ lies at the right end of the circle
  and $(0\mathrm{pt},1\mathrm{pt})$ lies at the top of the
  circle. (More precisely, these points will lie exactly on the
  middle of the radial line.)
 \end{key}
 Here is an examples of how to use this shape:
 \begin{codeexample}[]
  \tikz \node [generic circle ISO14617,
   circuit symbol size = width 5 height 5,
   /pgf/generic circle ISO14617/before background={
     \pgfpathmoveto{\pgfpoint{-0.5pt}{-0.5pt}}
     \pgfpathlineto{\pgfpoint{1pt}{0pt}}
     \pgfpathlineto{\pgfpoint{0pt}{1pt}}
     \pgfpathlineto{\pgfpoint{-0.5pt}{-0.5pt}}
     \pgfusepath{draw}
    },
   transform shape,
   draw] {};
 \end{codeexample}
\end{shape}

\begin{shape}{generic rectangle ISO14617}
 The same as |generic circle ISO14617|, only now it inherits from |rectangle pid|, and it allows for non-uniform scaling along the x and y-axis.
 This shape also allows you to use you to specify a path that should be added before the background path using
 the following key:
 \begin{key}{/pgf/generic rectangle ISO14617/before background=\meta{code}}
  When a node of shape |generic rectangle ISO14617| is created, the current
  setting of this key is used as the ``before background path.''
  This means that after the rectangle's background has been
  drawn/filled/whatever, the \meta{code} is executed.

  When the \meta{code} is executed, the coordinate system will have
  been transformed in such a way that the point
  $(1\mathrm{pt},0\mathrm{pt})$ lies at the right end of the rectangle
  and $(0\mathrm{pt},1\mathrm{pt})$ lies at the top of the
  rectangle.
 \end{key}
 Here is an examples of how to use this shape:
 \begin{codeexample}[]
  \tikz \node [generic rectangle ISO14617,
   circuit symbol size = width 5 height 5,
   /pgf/generic rectangle ISO14617/before background={
     \pgfpathmoveto{\pgfpoint{-0.5pt}{-0.5pt}}
     \pgfpathlineto{\pgfpoint{1pt}{0pt}}
     \pgfpathlineto{\pgfpoint{0pt}{1pt}}
     \pgfpathlineto{\pgfpoint{-0.5pt}{-0.5pt}}
     \pgfusepath{draw}
    },
   transform shape,
   draw] {};
 \end{codeexample}
 When the |circuit symbol size| key is changed the scale of the before background path is altered as well.
 \begin{codeexample}[]
  \tikz \node [generic rectangle ISO14617,
   circuit symbol size = width 15 height 5,
   /pgf/generic rectangle ISO14617/before background={
     \pgfpathmoveto{\pgfpoint{-0.5pt}{-0.5pt}}
     \pgfpathlineto{\pgfpoint{1pt}{0pt}}
     \pgfpathlineto{\pgfpoint{0pt}{1pt}}
     \pgfpathlineto{\pgfpoint{-0.5pt}{-0.5pt}}
     \pgfusepath{draw}
    },
   transform shape,
   draw] {};
 \end{codeexample}
\end{shape}

\begin{shape}{2531 ISO14617}
 A shape based on the |rectangle| with a top compartment, where the |input| and |output| anchors are placed at the top part of the rectangle. This shape is used for symbol 2531 which is a boiler/steam generator.
 \begin{codeexample}[]
  \begin{tikzpicture}
   \node [2531 ISO14617,
    circuit symbol size = width 20 height 20,
    shape example,
    draw] (s) {};
   \foreach \anchor/\placement in
    {center/above,
     north/above, south/below, east/left, west/right,
     north east/above, south east/below, south west/below, north west/above,
     input/left,output/right}
   \draw[shift=(s.\anchor)] plot[mark=x] coordinates{(0,0)}
   node[\placement] {\scriptsize\texttt{(s.\anchor)}};
  \end{tikzpicture}
 \end{codeexample}
\end{shape}

\begin{shape}{2518 ISO14617}
 A shape based on the |rectangle| with a top compartment, where the following anchors are defined |input in|, |output in|, |input out| and |output out|. These anchors represent the ingoing flow and the return flow for a regenerative pre-heater.
 \emph{ToDo:} \href{https://github.com/peer23peer/PIDcircuitTikZ/issues/1}{issue \#1}
 \begin{codeexample}[]
  \begin{tikzpicture}
   \node [2518 ISO14617,
    circuit symbol size = width 25 height 40,
    shape example,
    draw] (s) {};
   \foreach \anchor/\placement in
    {center/above,
     north/above, south/below, east/left, west/right,
     north east/above, south east/below, south west/below, north west/above,
     input in/left,output in/right, input out/left,output out/right}
   \draw[shift=(s.\anchor)] plot[mark=x] coordinates{(0,0)}
   node[\placement] {\scriptsize\texttt{(s.\anchor)}};
  \end{tikzpicture}
 \end{codeexample}
\end{shape}

\begin{shape}{2521 ISO14617}
 A shape based on the |rectangle| with a bottom compartment, where the following anchors are defined |input| and |output|. This shape is used for symbol 2521 which is a cooling tower.
 \begin{codeexample}[]
  \begin{tikzpicture}
   \node [2521 ISO14617,
    circuit symbol size = width 20 height 40,
    shape example,
    draw] (s) {};
   % \foreach \anchor/\placement in
   %  {center/above,
   %   north/above, south/below, east/left, west/right,
   %   north east/above, south east/below, south west/below, north west/above,
   %   input/left,output/right}
   % \draw[shift=(s.\anchor)] plot[mark=x] coordinates{(0,0)}
   % node[\placement] {\scriptsize\texttt{(s.\anchor)}};
  \end{tikzpicture}
 \end{codeexample}
\end{shape}

\begin{shape}{heat exchanger long ISO14617}
 A shape based on the |rectangle| with a top and a bottom compartment, where the following anchors are defined |input|, |output|, |refrigerant in| and |refrigerant out|. This shape is used for symbol |2511| and |2514| which heat exhangers.
 This shape allows the use for a custum background path using the following key:
 \begin{key}{/pgf/heat exchanger long ISO14617/before background=\meta{code}}
  When a node of shape |heat exchanger long ISO14617| is created, the current
  setting of this key is used as the ``before background path.''
  This means that after the shapes background has been
  drawn/filled/whatever, the \meta{code} is executed.

  When the \meta{code} is executed, the coordinate system will have
  been transformed in such a way that the point
  $(1\mathrm{pt},0\mathrm{pt})$ lies at the right end of the rectangle
  and $(0\mathrm{pt},1\mathrm{pt})$ lies at the top of the
  rectangle.
 \end{key}

 \begin{codeexample}[]
  \begin{tikzpicture}
   \node [heat exchanger long ISO14617,
    circuit symbol size = width 25 height 50,
    shape example,
    draw] (s) {};
   \foreach \anchor/\placement in
    {center/above,
     north/above, south/below, east/left, west/right,
     north east/above, south east/below, south west/below, north west/above,
     input/left,output/right,refrigerant in/left,refrigerant out/right}
   \draw[shift=(s.\anchor)] plot[mark=x] coordinates{(0,0)}
   node[\placement] {\scriptsize\texttt{(s.\anchor)}};
  \end{tikzpicture}
 \end{codeexample}
\end{shape}

\subsection{ISO-14617 P\&ID library}
\begin{tikzlibrary}{circuits.pid.ISO14617}
 This library provides graphics for \emph{Process \& Instrumentation Diagrams} according to the international standard ISO-14617.

 \noindent This library defines the following key:

 \begin{key}{/tikz/circuit pid ISO14617}
  This key should be passed as an option to a picture or a scope that contains a diagram. It will do some internal setups.
 \end{key}

 \begin{key}{/tikz/sensor}
  test \keyalias{/tikz/751}
  \begin{codeexample}[]
   \tikz[circuit pid ISO14617] \node [sensor={info=center:W}] {};
  \end{codeexample}
 \end{key}

\end{tikzlibrary}

\section{Example symbols}

\subsection{ISO-14617-9 Pumps, compressors and fans}

\tabulinesep=2pt
\begin{tabu}{l  l  l  X[l] }
 \emph{ISO name} & \emph{Base Key} & \emph{Modifiers} & \emph{Appearance} \\[.25em]
 \pidsymbolexample{pump}{/tikz/pump}{}{}
 \pidsymbolexample{displacement pump}{/tikz/pump}{displacement}{displacement}
 \pidsymbolexample{adjustable displacement pump}{/tikz/pump}{displacement, adjustable}{displacement, adjustable}
 \pidsymbolexample{centrifugal pump}{/tikz/pump}{centrifugal}{centrifugal}
 \pidsymbolexample{rotodynamic pump}{/tikz/pump}{rotodynamic}{rotodynamic}

 \pidsymbolexample{compressor}{/tikz/compressor}{}{}
 \pidsymbolexample{fan}{/tikz/fan}{}{}
\end{tabu}

\subsection{ISO-14617-11 Devices for heat transfer and heat engines}

\tabulinesep=2pt
\begin{tabu}{l  l  l  X[r] }
 \emph{ISO name} & \emph{Base Key} & \emph{Modifiers} & \emph{Appearance} \\[.25em]
 \pidsymbolexample{tank}{/tikz/tank}{}{}
 \pidsymbolexample{spray nozzle}{/tikz/spray nozzle}{}{}
 \pidsymbolexample{humidifier}{/tikz/tank}{with={spray nozzle}{0}{0.5}}{with=\{spray nozzle\}\{0\}\{0.5\}}
\end{tabu}

\printindex

%
% \section{Introduction}
% \begin{texexptitled}{Simple circuit}{Simple circuit}
%  \usetikzlibrary{circuits}
%  \usetikzlibrary{circuits.pid.ISO14617}
%  \usetikzlibrary{positioning,calc}
%
%  \centering
%  \begin{tikzpicture}[
%    circuit pid ISO14617,
%    every info/.style={font=\tiny}]
%
%   \draw (0,0) to [pump={displacement,name=P1,info=$P_1$}] (2,0)
%   to [branch={name=T1}] (2.5,0)
%   to [flow direction={speed=3}] (3,0)
%   to [valve={name=V1,info'=$V_{1}$}](4,0)
%   to [three way valve={globe, name=V2,info=below right:$V_2$}] ++(1,0)
%   to [tank={name=B1,with={heating coil}{0pt}{0pt}}] ++(1,0)
%   to [tank={name=F1, with={filter element}{0}{-0.5}, with={spray nozzle}{0}{0.8}}] ++ (1,0);
%   \draw (V2.south) to [pump={name=P2,info=$P_2$}] ++(0,-2)
%   to [measurement point={name=M1}] ++(-2,0)
%   to (\currentcoordinate -| T1)
%   to [valve={non return,info=$V_3$}] (T1);
%   \node[measurement device=local control room, at={M1.center}{1}, measure=P]{};
%   \node[turning actuator, at={V1.center}{1}]{};
%   \node[automatic operation, at={V2.center}{1}]{M};
%   \node[steam generator={with={fired type}{0}{-0.25},name=B2}, below=of F1] {};
%   \draw (B1-heating coil.south) to  (B1-heating coil.north |- B2.input)
%   to [valve, circuit symbol unit=3pt] (B2.input);
%   \draw (B1-heating coil.north) to ++(0, 0.5)
%   to ++(1,0);
%   \draw (F1-spray nozzle.north) to ++(0,0.15)
%   to [valve, circuit symbol unit=3pt] ++(1, 0);
%  \end{tikzpicture}
% \end{texexptitled}
%
% \newpage
% \section{Available symbols}
%
% \subsection{ISO 14617-1 General information and indexes}
% No Symbols in norm
%
% \subsection{ISO 14617-2 Symbols having general application}
% \begin{symboltitled}{Direction}\draw (0,0) to [flow direction] (2,0);\end{symboltitled}
% \begin{symboltitled}{Direction speed}\draw (0,0) to [flow direction={speed=1}] (2,0);\end{symboltitled}
% \begin{symboltitled}{Direction mass flow}\draw (0,0) to [flow direction={mass flow=1}] (2,0);\end{symboltitled}
% \begin{symboltitled}{Direction volume flow}\draw (0,0) to [flow direction={volume flow=1}] (2,0);\end{symboltitled}
% \begin{symboltitled}{Envelope}\draw (0,0) to [envelope] (2,0);\end{symboltitled}
% \begin{symboltitled}{Tank}\draw (0,0) to [tank] (2,0);\end{symboltitled}
%
% \subsection{ISO 14617-3 Connections and related devices}
% \begin{symboltitled}{Functional connection - 401}\draw (0,0) to (2,0);\end{symboltitled}
% \begin{symboltitled}{Pipeline, duct - 405}\draw (0,0) to (2,0);\end{symboltitled}
% \begin{symboltitled}{Joint of connections, branch - 501}\draw (0,0) to [branch] (2,0);\end{symboltitled}
% \begin{symboltitled}{T-branch - X504}
%  \draw (0,0) to [branch={name=B1}] (2,0);
%  \draw (B1) to ++(0,1);
% \end{symboltitled}
% \begin{symboltitled}{Joint of multiple functional connections - X506}
%  \draw (0,0) to [branch={name=B1}] (2,0);
%  \draw (B1) to ++(0,1);
%  \draw (B1) to ++(0,-1);
% \end{symboltitled}
%
% \subsection{ISO 14617-4 Actuators and related devices}
% \begin{symboltitled}{Manual actuator operated by turning - 685}\node [turning actuator] {};\end{symboltitled}
% \begin{symboltitled}{Manual actuated valve operated by turning - X685}
%  \draw (0,0) to [valve={name=V1}] (2,0);
%  \node [turning actuator, at={V1.center}{1}] {};
% \end{symboltitled}
% \begin{symboltitled}{Spring - 2002}\node [spring] {};\end{symboltitled}
%
%
% \subsection{ISO 14617-5 Measurement and control devices}
% None available at the moment, feel free to contribute!
%
% \subsection{ISO 14617-6 Measurement and control functions}
% \begin{symboltitled}{Point of measurement - 1011}\draw (0,0) to[measurement point] (2,0);\end{symboltitled}
% \begin{symboltitled}{Measurement device - 1041}\node [measurement device] {};\end{symboltitled}
% \begin{symboltitled}{Measurement device primary location in a central control room - 1101}\node [measurement device={central control room}] {};\end{symboltitled}
% \begin{symboltitled}{Measurement device primary location in a local control room - 1101}\node [measurement device={local control room}] {};\end{symboltitled}
% \begin{symboltitled}{Pressure measurement}
%  \draw (0,0) to[measurement point={name=M1}] (2,0);
%  \node [measurement device, at={M1.center}{}, measure=P] {};
% \end{symboltitled}
% \begin{symboltitled}{Temperature indication in central control room - X1075}
%  \draw (0,0) to[measurement point={name=M1}] (2,0);
%  \node [measurement device={central control room}, at={M1.center}{1}, measure=TI] {};
% \end{symboltitled}
%
% \begin{extrainfo}{Letter symbols for data processing functions}
%  \tabulinesep=1mm
%  \begin{tabu}{|l | c | X | X | X |}\hline
%        & Symbol & Measured or initiating variable & Modifier            & Function                               \\ \hline
%   1051 & A      &                                 &                     & Alarming                               \\ \hline
%   1052 & B      &                                 &                     & Displaying discrete state              \\ \hline
%   1053 & C      &                                 &                     & Controlling                            \\ \hline
%   1054 & D      & Density                         & Difference          &                                        \\ \hline
%   1055 & E      & Electric variable               &                     & Sensing                                \\ \hline
%   1056 & F      & Flow rate                       & Ratio, fraction     &                                        \\ \hline
%   1057 & G      & Gauge, position, length         &                     & Viewing                                \\ \hline
%   1058 & H      & Hand                            &                     &                                        \\ \hline
%   1059 & I      &                                 &                     & Indicating                             \\ \hline
%   1060 & J      & Power                           & Scanning            &                                        \\ \hline
%   1061 & K      & Time                            & Time rate of change &                                        \\ \hline
%   1062 & L      & Level                           &                     &                                        \\ \hline
%   1063 & M      & Moisture, humidity              & Momentarily         &                                        \\ \hline
%   1064 & N      & User's choice                   &                     & User's choice                          \\ \hline
%   1065 & O      & User's choice                   &                     &                                        \\ \hline
%   1066 & P      & Pressure, vacuum                &                     & Connection of test point               \\ \hline
%   1067 & Q      & Quality                         & Integral, total     & Integrating, summing                   \\ \hline
%   1068 & R      & Radiation                       &                     & Registering, recording                 \\ \hline
%   1069 & S      & Speed, frequency                &                     & Switching                              \\ \hline
%   1070 & T      & Temperature                     &                     & Transmitting                           \\ \hline
%   1071 & U      & Multi-variable                  &                     & Multi-function                         \\ \hline
%   1072 & V      & User's choice                   &                     & Impact on process by valve, pump, etc. \\ \hline
%   1073 & W      & Weight, force                   & Multiplying         &                                        \\ \hline
%   1074 & X      & Unclassified                    &                     & Unclassified                           \\ \hline
%   1075 & Y      & User's choice                   &                     & Converting, computing                  \\ \hline
%   1076 & Z      & Number of events, quantity      &                     & Emergency or safety acting             \\ \hline
%  \end{tabu}
% \end{extrainfo}
%
% \begin{extrainfo}{Letter codes for set values}
%  \begin{tabu}{| l | l | X |} \hline
%        & Symbol & set value      \\ \hline
%   1081 & H      & High           \\ \hline
%   1082 & HH     & Very high      \\ \hline
%   1083 & H2     & Very high      \\ \hline
%   1084 & HHH    & Extremely high \\ \hline
%   1085 & H3     & Extremely high \\ \hline
%   1086 & L      & Low            \\ \hline
%   1087 & LL     & Very low       \\ \hline
%   1088 & L2     & Very low       \\ \hline
%   1089 & LLL    & Extremely low  \\ \hline
%   1090 & L3     & Extremely low  \\ \hline
%   1091 & HL     & High or low    \\ \hline
%  \end{tabu}
% \end{extrainfo}
%
% \subsection{ISO 14617-7 Basic mechanical components}
% \begin{symboltitled}{Spray nozzle - 2037}\node [spray nozzle] {};\end{symboltitled}
% \begin{symboltitled}{Pressure vessel - 2062}\node [pressure vessel] {};\end{symboltitled}
% \begin{symboltitled}{Container, tank, cistern for atmospheric pressure - 2061}\node [tank atmospheric pressure] {};\end{symboltitled}
%
% \subsection{ISO 14617-8 Valves and dampers}
% \begin{symboltitled}{Two-way valve - 2101}\draw (0,0) to [valve] (2,0);\end{symboltitled}
% \begin{symboltitled}{Two-way valve non return - 2101, 2111}\draw (0,0) to [valve={non return}] (2,0);\end{symboltitled}
% \begin{symboltitled}{Two-way valve globe - 2101, 2121}\draw (0,0) to [valve={globe}] (2,0);\end{symboltitled}
% \begin{symboltitled}{Two-way valve safety function - 2101, 2112}\draw (0,0) to [valve={safety function}] (2,0);\end{symboltitled}
% \begin{symboltitled}{Two-way valve adjustable - 2101, 201}\draw (0,0) to [valve={adjustable}] (2,0);\end{symboltitled}
% \begin{symboltitled}{Two-way valve manual operation - 2101, 1021}
%  \draw (0,0) to [valve={name=V1}] (2,0);
%  \node [manual operation, at={V1.center}{1}]{};
% \end{symboltitled}
% \begin{symboltitled}{Two-way valve turning actuator - 2101, 1021}
%  \draw (0,0) to [valve={name=V1}] (2,0);
%  \node [turning actuator={at={V1.center}{1}}]{};
% \end{symboltitled}
% \begin{symboltitled}{Two-way valve automatic operation (electric motor type) - 2101, 685}
%  \draw (0,0) to [valve={name=V1}] (2,0);
%  \node [automatic operation, at={V1.center}{1}]{M};
% \end{symboltitled}
% \begin{symboltitled}{Angled two-way valve - 2102}
%  \node [angled valve={name=V1}] {};
%  \draw (V1.east) to ++(0.5,0);
%  \draw (V1.south) to ++(0,-0.5);
% \end{symboltitled}
% \begin{symboltitled}{Angled two-way globe safety valve with spring return - 2102}
%  \node [angled valve={globe, safety function, name=V1}] {};
%  \node [spring={info=$ p > \SI{10}{\bar} $}, at={V1.center}{0.5}] {};
%  \draw (V1.east) to ++(0.5,0);
%  \draw (V1.south) to ++(0,-0.5);
% \end{symboltitled}
% \begin{symboltitled}{Three-way valve - 2103}
%  \draw (0,0) to [three way valve={name=V1}] (2,0);
%  \draw (V1.south) to ++(0,-0.5);
% \end{symboltitled}
% \begin{symboltitled}{Three-way valve globe - 2103, 2121}
%  \draw (0,0) to [three way valve={globe, name=V1}] (2,0);
%  \draw (V1.south) to ++(0,-0.5);
% \end{symboltitled}
% \begin{symboltitled}{Three-way valve automatic operation (electric motor type) - 2103, 685}
%  \draw (0,0) to [three way valve={name=V1}] (2,0);
%  \node [automatic operation, at={V1.center}{1}]{M};
%  \draw (V1.south) to ++(0,-0.5);
% \end{symboltitled}
% \begin{symboltitled}{Four-way valve - 2104}
%  \draw (0,0) to [four way valve={name=V1}] (2,0);
%  \draw (V1.south) to ++(0,-0.5);
%  \draw (V1.north) to ++(0,0.5);
% \end{symboltitled}
% \begin{symboltitled}{Four-way valve automatic operation (electric motor type) - 2104, 685}
%  \draw (0,0) to [four way valve={name=V1}] (2,0);
%  \draw (V1.south) to ++(0,-0.5);
%  \draw (V1.north) to ++(0,0.5);
%  \node [automatic operation={name=V1C}] at ([yshift=10]V1.north west) {M};
%  \draw (V1C) to (V1.north west) to (V1.center);
% \end{symboltitled}
% \begin{symboltitled}{Two- or three-way damper - 2151}\draw (0,0) to [damper] (2,0);\end{symboltitled}
% \begin{symboltitled}{Two- or three-way damper - 2151}\draw (0,0) to [damper={safety function}] (2,0);\end{symboltitled}
%
% \subsection{ISO 14617-9 Pumps, compressors and fans}
% \begin{symboltitled}{Pump}\draw (0,0) to [pump] (2,0);\end{symboltitled}
% \begin{symboltitled}{Displacement Pump}\draw (0,0) to [pump=displacement] (2,0);\end{symboltitled}
% \begin{symboltitled}{Adjustable displacement pump}\draw (0,0) to [pump={displacement,adjustable}] (2,0);\end{symboltitled}
% \begin{symboltitled}{Rotodynamic Pump}\draw (0,0) to [pump=rotodynamic] (2,0);\end{symboltitled}
% \begin{symboltitled}{Centrifugal Pump}\draw (0,0) to [pump=centrifugal] (2,0);\end{symboltitled}
% \begin{symboltitled}{Compressor}\draw (0,0) to [compressor] (2,0);\end{symboltitled}
% \begin{symboltitled}{Fan}\draw (0,0) to [fan] (2,0);\end{symboltitled}
%
% \subsection{ISO 14617-10 Fluid power converters}
% None available at the moment, feel free to contribute!
%
% \subsection{ISO 14617-11 Devices for heat transfer and heat engines}
% \begin{symboltitled}{Tank - 301}\draw (0,0) to [tank] (2,0);\end{symboltitled}
% \begin{symboltitled}{Spray nozzle - 2037}\node [spray nozzle] {};\end{symboltitled}
% \begin{symboltitled}{Humidifier - X2503, 301, 2037}
%  \draw (0,0) to [tank={with={spray nozzle}{0}{0.5},name=HU1}] (2,0);
%  \draw (HU1-spray nozzle.north) to ++(0, 0.5);
% \end{symboltitled}
% \begin{symboltitled}{Heating coil - 2501}\node [heating coil] {};\end{symboltitled}
% \begin{symboltitled}{Heat-exchanger - X2501, 301, 2501}
%  \draw (0,0) to [tank={with={heating coil}{0}{0},name=HE1}] (2,0);
%  \draw (HE1-heating coil.north) to ++(0, 0.5);
%  \draw (HE1-heating coil.south) to ++(0, -0.5);
% \end{symboltitled}
% \begin{symboltitled}{Cooling coil - 2501}\node [cooling coil] {};\end{symboltitled}
% \begin{symboltitled}{Condensor - X2501, 301, 2501}
%  \draw (0,0) to [tank={with={cooling coil}{0}{0},name=CO1}] (2,0);
%  \draw (CO1-cooling coil.north) to ++(0, 0.5);
%  \draw (CO1-cooling coil.south) to ++(0, -0.5);
% \end{symboltitled}
% \begin{symboltitled}{Boiler - 301, 2531}\node [boiler] {};\end{symboltitled}
% \begin{symboltitled}{Fired type - 2541}\node [fired type] {};\end{symboltitled}
% \begin{symboltitled}{Boiler - 301, 2531}\node [boiler={with={fired type}{0}{-0.5}}] {};\end{symboltitled}
% \begin{symboltitled}{Boiler example - 301, 2531}
%  \node[boiler={with={fired type}{0}{-0.25},name=B1}] {};
%  \draw (B1.output) to ++(0.5,0);
%  \draw (B1.input) to ++(-0.5,0);
%  \draw (B1-fired type.south) to ++(0,-0.5)
%  to ++(0.5,0);
% \end{symboltitled}
% \begin{symboltitled}{Steam generator - 301, 2531}\node [steam generator] {};\end{symboltitled}
% \begin{symboltitled}{Steam generator with heating coil - 301, 2531, 2501}
%  \node [steam generator={with={heating coil}{0}{-0.25}, name=SG1}] {};
%  \draw (SG1.input) to ++(-0.5,0);
%  \draw (SG1.output) to ++(0.5,0);
%  \draw (SG1-heating coil.north) to ++(0.75,0);
%  \draw (SG1-heating coil.south) to ++(0.75,0);
% \end{symboltitled}
% \begin{symboltitled}{Heat pump - 2551, 130}\draw (0,0) to [heat pump={quantity transition to higher}] (2,0);\end{symboltitled}
% \begin{symboltitled}{Heat pump bi-functional - 2551, 132}\draw (0,0) to [heat pump={quantity transition to higher and lower}] (2,0);\end{symboltitled}
% \begin{symboltitled}{Refrigerator - 2551, 131}\draw (0,0) to [refrigerator={quantity transition to lower}] (2,0);\end{symboltitled}
% \begin{symboltitled}{Freezer - 2551, 131}\draw (0,0) to [freezer={quantity transition to lower}] (2,0);\end{symboltitled}
% \begin{symboltitled}{Regenerative pre-heater - 2518}
%  \node [regenerative pre-heater={name=PH1}] {};
%  \draw (PH1.input in) to ++(-0.5,0);
%  \draw (PH1.output in) to ++(0.5,0);
%  \draw (PH1.input out) to ++(-0.5,0);
%  \draw (PH1.output out) to ++(0.5,0);
% \end{symboltitled}
% \begin{symboltitled}{Heat-exchanger with straight tubes (fixed-tube plates) - 2511}
%  \node [straight tube heat exchanger={name=PH1}] {};
%  \draw (PH1.input) to ++(-0.5,0);
%  \draw (PH1.output) to ++(0.5,0);
%  \draw (PH1.refrigerant in) to ++(-0.5,0);
%  \draw (PH1.refrigerant out) to ++(0.5,0);
% \end{symboltitled}
% \begin{symboltitled}{Cooling tower - 2521}
%  \node [cooling tower={name=CT1}] {};
% \end{symboltitled}
% \begin{symboltitled}{Cooling tower, humidifier - X2521}
%  \node [cooling tower={name=CT1, with={spray nozzle}{0}{2.5}}] {};
%  \draw (CT1-spray nozzle.north) to ++(-1,0);
% \end{symboltitled}
% \begin{symboltitled}{Cooling tower with induced draft - X2522}
%  \node [cooling tower={with={fan}{0}{2.5}{rotate=90, scale=0.5}}] {};
% \end{symboltitled}
% \begin{symboltitled}{Cooling tower with forced draft - X2523}
%  \node [cooling tower={with={fan}{0}{-3}{rotate=90, scale=0.5}}] {};
% \end{symboltitled}
% \begin{symboltitled}{Solar collector}
%  \node [envelope={light, quantity transition to higher, info={[xshift=-2, yshift=-10]\tiny T}}] {};
% \end{symboltitled}
%
% \subsection{ISO 14617-12 Devices for separating, purification and mixing}
% \begin{symboltitled}{Tank - 301}\draw (0,0) to [tank] (2,0);\end{symboltitled}
% \begin{symboltitled}{Spray nozzle - 2037}\node [spray nozzle] {};\end{symboltitled}
% \begin{symboltitled}{Screen element - 2602}\node [screen element] {};\end{symboltitled}
% \begin{symboltitled}{Filter - X2601 301, 2602}\node [tank={with={filter element}{0}{-0.25}}] {};\end{symboltitled}
% \begin{symboltitled}{Filter with spray - X2607 301, 2037, 2602}
%  \node [tank={with={filter element}{0}{-0.5}, with={spray nozzle}{0}{0.5}, name=WA1}] {};
%  \draw (WA1-spray nozzle.north) to ++(0,0.25)
%  to ++(0.5,0);
%  \draw (WA1.input) to ++(-0.5,0);
%  \draw (WA1.output) to ++(0.5,0);
% \end{symboltitled}
% \begin{symboltitled}{Filter element - 2602}\node [filter element] {};\end{symboltitled}
% \begin{symboltitled}{Bag filter element - 2602-bag}\node [bag filter element] {};\end{symboltitled}
% \begin{symboltitled}{Bag filter - X2606 301, 2602}\node [tank={with={bag filter element}{0}{0.25}}] {};\end{symboltitled}
% \begin{symboltitled}{Device for separating - 2601}\node [device for separating] {};\end{symboltitled}
% \begin{symboltitled}{Thermal separator using direct-heating source - X2632 2541, 2601}\node [envelope={device for separating, with={fired type}{0}{-0.5}}] {};\end{symboltitled}
% \begin{symboltitled}{Ion exchange separator - X2633}\node [envelope={device for separating, info={[yshift=-15]\tiny ION}}] {};\end{symboltitled}
% \begin{symboltitled}{Biological filter - X2634}\node [envelope={device for separating, info={[yshift=-15]\tiny BIO}}] {};\end{symboltitled}
% \begin{symboltitled}{Chemical filter - X2632}\node [envelope={device for separating, info={[yshift=-15]\tiny CH}}] {};\end{symboltitled}
% \begin{symboltitled}{Rotary mixing element - 2672}\node [rotary mixing element] {};\end{symboltitled}
% \begin{symboltitled}{Mixer stirrer - X2671}
%  \node [envelope={with={rotary mixing element}{0}{-0.3}, name=M1}] {};
%  \draw [shaft] (M1-rotary mixing element.center) to ++(0,1);
% \end{symboltitled}
%
% \subsection{ISO 14617-13 Devices for material processing}
% None available at the moment, feel free to contribute!
%
% \subsection{ISO 14617-14 Devices for transport and handling of material}
% \begin{symboltitled}{Slope - 3061}\draw [slope] (0,0) to (2,0);\end{symboltitled}
% \begin{symboltitled}{Slope - 3061}\draw [slope, pos slope=0.25] (0,0) to (2,0);\end{symboltitled}
% \begin{symboltitled}{Slope - 3061}\draw [slope'] (0,0) to (2,0);\end{symboltitled}
%
%
% \subsection{ISO 14617-15 Installation diagrams and network maps}
% None available at the moment, feel free to contribute!

\end{document}
